\documentclass[a4paper]{article}
\usepackage[utf8]{inputenc}
\usepackage{amsmath}
\usepackage{amssymb}
\usepackage{physics}
\usepackage[margin=1cm]{geometry}
\usepackage{url}
\usepackage{xspace}
\usepackage{hyperref}

\setlength{\parskip}{1em}
\setlength{\parindent}{0em}

%\title{AP axis alignment in \Ce embryos}
\title{Elucidating the mechanism of AP axis alignment in the \textit{C. elegans} embryo}
\author{Archit Bhatnagar}
\date{}

\begin{document}

\maketitle 
\thispagestyle{empty}

Development of a single-cell embryo into an adult multi-cellular organism features the establishment of upto three anatomical body axes - anteroposterior, dorsoventral and left-right. It has been observed in many organisms that these body axes can consistently orient relative with respect to the geometric features of the embryo in many organisms. One such example is observed in the model organism \textit{Caenorhabditis elegans} (\textit{C. elegans}), where the anteroposterior (AP) axis coincides with the geometric long axis of the ellipsoidal embryo -- the shape being imposed by the surrounding eggshell. In \textit{C. elegans}, the AP axis is established at the one-cell stage via its polarization by PAR polarity proteins. This cell polarization proceeds via a self-organized mechanochemical feedback between the PAR proteins and mechanical flows in the actomyosin cortex, resulting in the formation of two mutually exclusive domains of aPAR and pPAR proteins on the cortex denoting the future anterior and posterior end of the embryo -- and thus establishing the AP axis. The initial orientation of the AP axis is determined by the site of sperm entry at fertilization. However, the nascent AP axis that forms after fertilization is observed to actively re-orient -- indicated by the movement of the PAR domains and concurrent migration (here termed posteriorisation) of the sperm-donated male pronucleus -- such that it aligns with the long axis of the ellipsoidal embryo, if it is not already aligned. In effect, the site of sperm entry only determines which half of the embryo becomes the posterior half of the embryo. This phenomenon of active re-orientation of the AP axis, that ensures that the AP axis aligns with the long axis of the embryo, is termed AP axis alignment. The work described in this thesis investigates the mechanism of this AP axis alignment in the \textit{C. elegans} embryo.

Anterior-directed flows in the actomyosin cortex observed during AP axis establishment have also been found to be essential for AP axis alignment. In this thesis, two possible mechanisms of AP axis alignment are considered, both of which are consequences of these cortical flows. Cortical flows at the embryo surface can drive flows in the bulk cytoplasm in the embryo, generating cytoplasmic flows which point towards the sperm-donated male pronucleus as it posteriorises. Previous studies have proposed that these cytoplasmic flows could push onto the male pronucleus, and due to the ellipsoidal geometry of the embryo, drive it towards the closest tip of the embryo. This proposed mechanism is referred to as the cytoplasmic flow-dependent mechanism in this thesis. Another mechanism proposed in this thesis postulates that the reorientation of the AP axis occurs via the repositioning of the pseudocleavage furrow. The pseudocleavage furrow is a contractile ring-like structure that forms at the boundary of the two PAR domains during AP axis establishment. The pseudocleavage furrow forms as a result of compressive alignment of actin filaments in the actomyosin cortex due to cortical flows. In cases where the AP axis is not aligned with the long axis of the embryo, the pseudocleavage furrow is
not perpendicular to the long axis of the embryo. In such cases, active anisotropic stresses generated in the actomyosin cortex could force the rotation of the pseudocleavage furrow akin to an elastic rubber-band on an ellipsoid, and cause the AP axis to re-orient towards the long axis of the embryo. This proposed mechanism is referred to as the pseudocleavage furrow-dependent mechanism in this thesis.

This thesis investigates the role played by the two mechanisms in AP axis alignment. This is accomplished in the following way: a theoretical model of the AP axis alignment is introduced, consisting of a description of the actomyosin cortex as an active nematic fluid present on the 2D surface of a fixed ellipsoid representing the embryo. This description of the cortex incorporates both the cytoplasmic flow-dependent mechanism and the pseudocleavage furrow-dependent mechanism. RNAi experiments in the \textit{C. elegans} embryo that remove the pseudocleavage furrow, in conjuction with numerical simulations using the theoretical model, show that the pseudocleavage furrow-dependent mechanism is the predominant mechanism that drives AP axis alignment, while cytoplasmic flow-dependent mechanism plays only a minor role. RNAi experiments that modify the geometry of the \textit{C. elegans} embryo -- specifically, generate rounder embyros -- show that embryo geometry can influence the rate of re-orientation of the AP axis during AP axis alignment -- with slower AP axis alignment in rounder embryos. Such an relation between embryo geometry and AP axis alignment is found to be consistent with pseudocleavage furrow-dependent mechanism, both via predictions made using the theoretical model and using a simplified effective model of a contractile ring (or elastic rubber-band) on a fixed ellipsoid. 

Altogether, the work presented in this thesis shows AP axis alignment observed in the \textit{C. elegans} embryo is driven primarily by the anisotropic stresses in the actomyosin cortex that generate the pseudocleavage furrow. The work here also shows that the AP axis alignment process is sensitive to the geometry of the embryo. In effect, active mechanical flows in the actomyosin cortex translate the ellipsoidal geometry of the embryo into a robust orientation of the AP axis of the \textit{C. elegans} embryo. Mechanical flows such as these are not exclusive to \textit{C. elegans}, nor are specific orientations of the body axes with respect to the embryo geometry. The results in this thesis thus point towards a possibly general role of the interactions between mechanical flows and embryo geometry to properly orient the body axes of the developing embryos of many multi-cellular organisms.

\end{document}