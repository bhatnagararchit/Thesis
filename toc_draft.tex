\documentclass{article}
\usepackage[utf8]{inputenc}
\usepackage{amsmath}
\usepackage{amssymb}
\usepackage{physics}
\usepackage[margin=1cm]{geometry}
\usepackage{url}
\usepackage{xspace}

\setlength{\parskip}{1em}
\setlength{\parindent}{0em}

\newcommand{\Ce}{\textit{C.~elegans}\xspace}

%\title{AP axis alignment in \Ce embryos}
\title{Orienting and building the actomyosin cortex in \Ce}
\author{Archit Bhatnagar}
\date{January 2023}

\begin{document}

\maketitle 

\section*{Table of Contents}

\begin{itemize}
    \item Part 1: AP axis alignment in \Ce [Set down the aim of the project here]
    \begin{itemize}
    \item Introduction [Describe general problem of axis selection, and geometry features]
    \begin{itemize}
            \item \Ce as a model organism
            \item AP axis establishment in \Ce
            \begin{itemize}
                \item AP axis is set by polarisation of one-cell stage
                \item Flows in the actomyosin cortex facilitate polarization
            \end{itemize}
            \item Active Matter theory: a review [Look to Sebastian's thesis, De Maur]
            \begin{itemize}
                \item Navier Stokes
                \item Entropy production and onsanger relations
                \item Nematics
                \item Actomyosin cortex as an Active nematic fluid
            \end{itemize}
            \item How geometry can influence development [Not sure if this is needed - maybe covered in main introduction]
        \end{itemize}
    \item Methods
    \begin{itemize}
        \item Experiments
        \begin{itemize}
            \item Experimental setup
            \item Measuring the movement of the male pronucleus
            \item Measuring cortical flows using PIV
            \item Genetic perturbations via RNAi
        \end{itemize}
        \item Mathematical model
        \begin{itemize}
            \item Cortex as an active isotropic fluid
            \item Cortex as an active nematic fluid
        \end{itemize}
    \end{itemize}
    \item Results
    \begin{itemize}
        \item Cortical flows are required for axis alignment (mlc-4)
        \item Pseudocleavage furrow is predominant in axis alignment (nop-1, nop-1/mel-11)
        \item Embryo shape influences axis alignment (ima-3, (maybe dpy-11, dpy-11/ima-3))
        \item Minimal model of contractile ring on an ellipsoid
    \end{itemize}
    \item Discussion
    \end{itemize}
    \item Part 2: Hamiltonian dynamics in Lotka-Volterra-Like systems
    \begin{itemize}
        \item Introduction
        \begin{itemize}
            \item Cortical Condensates [Discuss Actin-Wsp condensdates]
            \item Hamiltonian dynamics of Lotka-Volterra systems [Lotka-Volterra - derive hamiltonian, statistical mechanics discussion?]
        \end{itemize}
        \item Hamiltonian dynamics of cortical condensates
        \item Possible applications [Illustrate using Statistical Mechanics construction and comparison with data, make note of liouville only needed?]
    \end{itemize}
\end{itemize}

\end{document}