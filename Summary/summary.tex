In the \ac{ce} embryo, the \ac{ap} axis is established at the one-cell stage. It has been observed that the \ac{ap} axis always forms along the long axis of the embryo, and that the former re-orients to align with the latter \citep{goldstein1996specification}. The aim of the work presented in this thesis was to elucidate the mechanism behind this \ac{ap} axis alignment in the \ac{ce} embryo. Two possible mechanisms of \ac{ap} axis alignment, arising as a consequence of flows in the actomyosin cortex of the \ac{ce} embryos, were considered in this thesis (as depicted in \autoref{fig:apAxisAlignment}):
\begin{description}
    \item[Cytoplasmic flow-dependent Mechanism]\hfill\\
    Cortical flows at the embryo surface drive flows in the bulk cytoplasm \citep{niwayama2011hydrodynamic}. The cytoplasmic flows thus generated have been observed to be directed towards the male pronucleus as it posteriorises \citep{goldstein1996specification}. It was proposed that these flows could push onto the male pronucleus \citep{kimuraCytoplasmicFlows}, and owning to the geometry of the embryo in which the flows operate, push the male pronucleus towards the closest tip \citep{goldstein1996specification}. This mechanism was proposed in \citep{goldstein1996specification}.
    \item[Pseudocleavage furrow-dependent Mechanism]\hfill\\
    Cortical flows also lead to the formation of the pseudocleavage furrow by compressive alignment of actin filaments in the cortex. The pseudocleavage furrow is a contractile ring-like structure that forms at the boundary of the \ac{par} that specify the \ac{ap} axis -- and thus perpendicular to the instanteous \ac{ap} axis. It was proposed in this thesis that the rotation of the pseudocleavage furrow as it minimizes its circumference would drive the rotation of the cortex and re-orient the \ac{ap} axis towards the long axis of embryo. 
\end{description}

In \autoref{ch:ActiveMatter}, a theoretical model of \ac{ap} axis alignment was described that incorporates these two mechanisms. This quasi-steady state theoretical model consists of the description of the cortex as an active nematic fluid on the surface of a fixed ellipsoid that represents the embryo, the description of the cytoplasm as a Newtonian fluid filling the bulk of the ellipsoid (following \cite{niwayama2011hydrodynamic}) and the description of the transport of the male pronucleus via advection by cytoplasmic flows and drag with the cortex. The description of the cortex in the theoretical model is constructed by integrating elements of the descriptions of the cortex used in \cite{gross2019guiding} and \cite{reymann2016cortical} into a general hydrodynamic theory of active compressible nematic fluids \citep{julicher2018hydrodynamic}, and converted to a suitable surface description using the thin film limit \citep{nitschke2018nematic}. The two mechanisms listed above are incorporated via two different active stresses in this description of the cortex, as shown in \autoref{eq:dynamicsCortexVelocitySurfaceModelMN} for the cortical flow velocity $v$ (reproduced here):
\begin{multline*}\label{eq:dynamicsCortexVelocitySurfaceModelMNReproduceSummary}
    -\lambda_H\left[\divSurf\gradSurf\vec{v} + \mathcal{K}\vec{v} + \frac{1}{3}\gradSurf\divSurf\vec{v}\right] + \vec{v} \\= \lambda_A\gradSurf\left(\frac{M}{M + M_*}\right) + \lambda_N \gradSurf\left[\left(\frac{M}{M + M_*}\right)\left(\frac{1}{2\sqrt{2}}\norm{\mathbf{Q}}\mathbf{g} + \frac{1}{2}\mathbf{Q}\right)\right]    
\end{multline*}
where $M$ is the myosin concentration \citep{gross2019guiding}, $Q$ characterises the local alignment of actin filaments \citep{reymann2016cortical} and the parameters \hydrodynamicLength, \activeRelaxLength and \nematicLength characterise the passive and active stresses in the cortex relative to frictional drag with respect to the eggshell and cytoplasm -- with \hydrodynamicLength characterising the passive viscous stress, \activeRelaxLength characterising the active isotropic stress and \nematicLength characterising the active anisotropic stress generated by the local alignment of actin filaments (see \autoref{sec:apAxisAlignmentModelMN} for details). In the cytoplasmic flow-dependent mechanism, only the active isotropic stress characterised by \activeRelaxLength is considered \citep{gross2019guiding,mayer2010anisotropies}; while in the pseudocleavage furrow-dependent mechanism, the active anisotropic stress is also considered \citep{reymann2016cortical}. In effect, the strength of the pseudocleavage furrow-dependent mechanism is characterised by \nematicLength. Note that the active isotropic stress is generated solely by the action of myosin motors in the cortex, while the active anisotropic stress is generated as a result of local alignment of actin filaments and their interaction with the myosin motors. This active anisotropic stress is responsible for the contractile nature of the pseudocleavage furrow \citep{reymann2016cortical}. Such a distinction effectively implies two different descriptions of the cortex in the two mechanisms of \ac{ap} axis alignment considered here: the cytoplasmic flow-dependent mechanism effectively considers an active isotropic cortex with only myosin motors as relevant for force generation, while the pseudocleavage furrow-dependent mechanism considers the full active nematic description with both myosin motors and actin filaments relevant for force generation in the cortex. These three parameters, along with \dragCoefficient that characterises the cortical drag on the male pronucleus (see \autoref{eq:pronucleusTransportModelMN}) comprise the full set of model parameters that need to be determined for numerical simulations of the theoretical model. These parameters are determined by calibrating the model with experimentally observed cortical flows, as described in \autoref{subsec:numericsModelMN} and \autoref{subsec:expVsTheoryPcFurrow}.

The experimental methods and materials are detailed in \autoref{ch:Exp}. In particular, the image analysis used to track the male pronucleus as it migrates towards the posterior end (that is, the closest tip of the embryo) is described. This migration of the male pronucleus, termed posteriorisation, is used as the readout of the \ac{ap} axis alignment -- given the role of the male pronucleus as an organizer of \ac{ap} axis establishment via the centrosomes associated with it. Specifically, the \enquote{Posteriorisation velocity} of the male pronucleus -- the component of its velocity that is parallel to the cortex is obtained as a function of the \enquote{Angular position} of the male pronucleus -- the angle between the long axis and the line connecting the center of the male pronucleus to the center of the embryo.  

The experimental results and their comparisons to numerical simulations are detailed in \autoref{ch:Results}. In \autoref{sec:apAxisAlignCharacteriseWT}, it is observed in unperturbed embryos that the posteriorisation velocity is faster at higher angular positions, indicating that the rate of re-orientation of the \ac{ap} axis increases the further away the male pronucleus is from the posterior pole. In \autoref{sec:corticalFlowsRoleMlc4}, embryos with reduced cortical flows generated using \geneExp{mlc-4} \ac{rnai} show highly diminished posteriorisation velocity of the male pronucleus at all angular positions compared to unperturbed embryos, and a lack of posteriorisation of the male pronucleus -- demonstrating that cortical flows are essential for \ac{ap} axis alignment. 

In \autoref{sec:PcFurrowRole}, the role of the pseudocleavage furrow in \ac{ap} axis alignment is investigated. Embryos deficient in a pseudocleavage furrow, but with cortical flows comparable to unperturbed embryos, are generated using a double \ac{rnai} of \geneExp{nop-1; mel-11}. These pseudocleavage furrow-deficient embryos exhibit slower posteriorisation velocity compared to unperturbed embryos (which do have a pseudocleavage furrow) for all angular positions of the male pronucleus observed. However, unlike the \geneExp{mlc-4} \ac{rnai} embryos, these pseudocleavage furrow-deficient embryos do exhibit posteriorisation of the male pronucleus, and thus \ac{ap} axis alignment -- albeit at a slower rate compared to unperturbed embryos. These experimental observations indicate a role for the pseudocleavage furrow in the dynamics of \ac{ap} axis alignment. To further investigate the role of the pseudocleavage furrow -- and specifically to understand the relative contributions of the pseudocleavage furrow-dependent mechanism and cytoplasmic flow-dependent mechanism -- these experimental results are compared to numerical simulations of the theoretical model introduced in \autoref{ch:ActiveMatter}. Numerical simulations of the full theoretical model, including both the cytoplasmic flow-dependent and pseudocleavage furrow-dependent mechanisms and calibrated using experimentally observed cortical flows in the unperturbed embryos (yielding \hydrodynamicLength = \SI{10}{\unitLength}, \activeRelaxLength = \SI{11.5}{\square\unitLength\per\second}, \nematicLength = \SI{152.5}{\square\unitLength\per\second}), recapitulate the observed posteriorisation velocity in unperturbed embryos for a fitted value of \dragCoefficient = \num{0.61}, for angular positions in \SIrange{0}{21}{\unitAngle}. Numerical simulations of the theoretical model lacking in the pseudocleavage furrow-dependent mechanism (setting \nematicLength = \SI{0}{\square\unitLength\per\second} and calibrated using experimentally observed cortical flows in the \geneExp{nop-1; mel-11} \ac{rnai} embryos (yielding \hydrodynamicLength = \SI{11}{\unitLength}, \activeRelaxLength = \SI{11.5}{\square\unitLength\per\second}) with \dragCoefficient = \num{0.61} can capture the observed posteriorisation velocity in pseudocleavage furrow-deficient embryos for angular positions in \SIrange{0}{21}{\unitAngle}. Allowing \nematicLength to vary for the calibration with \geneExp{nop-1; mel-11} \ac{rnai} embryos still yields a smaller value of \nematicLength = \SI{25}{\square\unitLength\per\second} compared to the unperturbed embryos \nematicLength = \SI{152.5}{\square\unitLength\per\second}, while recapitulating the observed posteriorisation velocity in the pseudocleavage furrow-deficient embryos, for angular positions in \SIrange{0}{21}{\unitAngle}. Altogether, these observations -- both in experiments and in numerical simulations -- demonstrate that the pseudocleavage furrow-dependent mechanism is the predominant mechanism of \ac{ap} axis alignment in unperturbed embryos, with a minor role played by the cytoplasmic flow-dependent mechanism that is evident in the pseudocleavage furrow-deficient embryos.

How does this square with the previous observations that the pseudocleavage furrow is largely dispensable for proper \ac{ap} axis formation \citep{rose1995pseudocleavage, cowan2004asymmetric}? First, due to the geometrical constraints on the embryo during fertilisation, the sperm entry site is typically on the future posterior tip of embryo \citep{goldstein1996specification}. Thus, in the typical case, \ac{ap} axis is already aligned with the long axis of the embryo. Second, while the pseudocleavage furrow plays a predominant role in \ac{ap} axis alignment, the slower \ac{ap} axis alignment driven by cytoplasmic flows can still correct for small deviations of the \ac{ap} axis from the long axis. Third, incomplete \ac{ap} axis alignment at the establishment phase -- which has been the focus of this thesis -- can be corrected later by slow movement of \ac{par} domains in a mechanism independent of flows in the actomyosin cortex \citep{mittasch2018non,gessele2020geometric}. Thus, multiple redundant mechanisms ensure proper positioning of the \ac{ap} axis even if the pseudocleavage furrow fails to form.

Cytoplasmic flows have been observed as a general mechanism for repositioning of various structures in the cytoplasm in many biological systems, such as chloroplasts in \emph{Chara corallina} \citep{goldstein2015physical}, cytoplasmic components and nuclei in \emph{Drosophila} \citep{quinlan2016cytoplasmic,deneke2019self}, and spindle in the human and mouse oocytes \citep{yi2011dynamic,wang2020symmetry}. In systems such as \ac{ce} embryo, \emph{Drosophila}, human and mouse oocytes, cortical flows drive these cytoplasmic flows \citep{niwayama2011hydrodynamic,yi2011dynamic,deneke2019self,wang2020symmetry}. The work presented here demonstrates another mechanism by which cortical flows can influence the positioning of internal structures of the cell, via the repositioning of a contractile ring -- the pseudocleavage furrow. Such a contractile ring arises in the \ac{ce} cortex as a result of the nematic ordering in the cortex conferred on it by the actin filaments \citep{reymann2016cortical}. Given the importance of the actomyosin cortex in development in multiple different organisms \citep{gross2017active}, it would be interesting to study the effect of such properties of the cortex on the positioning of structures in the cytoplasm in different organisms.

In \autoref{sec:GeometryRole}, the role of embryo geometry in \ac{ap} axis alignment in \ac{ce} embryos is investigated. Reducing the aspect ratio of the ellipsoid used in the numerical simulations of the full theoretical model predicted that \ac{ap} axis alignment should be slower in rounder embryos (embryos with a smaller aspect ratio compared to unperturbed embryos). To test this prediction, rounder embryos were generated using \ac{rnai} of \geneExp{ima-3}. It is observed that the theoretical model simulated using an ellipsoid with the average aspect ratio of these rounder embryos can recapitulate the experimentally observed posteriorisation velocity in these rounder embryos for the same set of model parameters used for the unperturbed embryos. That is, experimental observations confirm the predictions from the numerical simulations: slower \ac{ap} axis alignment observed in rounder embryos is quantitatively consistent with the predictions made by the theoretical model. Interestingly, the change in volume between the rounder embryos and unperturbed embryos is found to be not important for the difference in \ac{ap} axis alignment between the two sets of embryos. As previous results indicated a predominant role for the pseudocleavage furrow in \ac{ap} axis alignment, an effective model of a contractile ring that slips on the surface of a fixed ellipsoid is proposed to mimic the repositioning of the pseudocleavage furrow during \ac{ap} axis alignment. Such an effective model -- comprising only of two features of the full theoretical model: ellipsoidal geometry and contractile ring -- is found to capture the relation between embryo geometry and \ac{ap} axis alignment. Altogether, these observations -- both in experiments and in numerical simulations -- demonstrate that \ac{ap} axis alignment is sensitive to the geometry of the \ac{ce} embryo, in a manner that is captured by the pseudocleavage furrow-dependent mechanism. This further confirms the predominant role of the pseudocleavage furrow in \ac{ap} axis alignment in \ac{ce} embryos.

Altogether, the work presented in this thesis shows that \ac{ap} axis alignment in the \ac{ce} embryo is driven by active mechanical flows in the actomyosin cortex that generate two distinct mechanisms for \ac{ap} axis alignment: pseudocleavage furrow-dependent mechanism and cytoplasmic flow-dependent mechanism. The pseudocleavage furrow-dependent mechanism is the predominant mechanism of \ac{ap} axis alignment, and arises as a consequence of the active anisotropic stresses in the actomyosin cortex due to alignment of actin filaments in the cortex. The cytoplasmic flow-dependent mechanism is a subsidiary mechanism, and arises as a consequence of active isotropic stresses in the cortex due to action of myosin motors alone. Furthermore, embryo geometry is shown to have an influence on \ac{ap} axis alignment in \ac{ce} embryos. Such a relation betwen the two is also shown to be consistent with this predominant role of the pseudocleavage furrow in \ac{ap} axis alignment in \ac{ce} embryos.

Many of the processes involved in the establishment of \ac{ap} axis are not unique to the \ac{ce} embryos. Body axes establishment is often mediated by self-organised pattern formation processes \citep{GerhartKirschnerCells,goldstein1997axis}, with important role for mechanical forces \citep{sagyMechanicsOrganoid,etoc2016balance,zhang2019mouse,gross2017active,munster2018integrin} -- such as the mechanochemical feedback between \ac{par} proteins and myosin motors in the actomyosin cortex that establish the \ac{ap} axis in \ac{ce} embryo \citep{gross2019guiding}. Additionally, geometric features in the embryo have been observed to direct the orientation of body axis during development -- as reviewed in \autoref{ch:APAxisIntro}. The work on \ac{ap} axis alignment in \ac{ce} embryos presented in this thesis indicates that the interplay between mechanical forces in the embryo and such geometric features may robustly orient the body axes relative to the geometry of the embryo. It would be interesting to investigate if such an interplay between mechanics and geometry is a general feature of body axes establishment in development of multi-cellular organisms.

%WHATS THE MESSAGE/Q AT THE END? EITHER END ON A GENERAL STATEMENT OR A CONCISE PROPOSAL WHAT TO PERHAPS DO NEXT.
%RETHINK THE LAST 2 SENTENCES.. BETTER: SENTENCE ON GUIDANCE BY GEOMETRY FIX SENTENCE: Our work suggests that such interactions (e.g. formation of a contractile ring \citep{litschel2021reconstitution,salbreux2009rings}) plays a role in the guidance of body axes establishment by geometric features. Elucidating the different processes of geometric guidance during body axes establishment, and what role of mechanical flows play in these processes, would be an interesting subject for future work.

% Pseudocleavage furrow dependent mechanism role
% To cover: PC is dispensible, but important for proper dynamics. Other processes - Maintenance phase PAR domain stabilization - can still rectify improper PAR domain establishment later on
%We reveal a novel function for the pseudocleavage furrow in proper AP axis establishment in \ac{ce} embryo.  
%How could a mechanism that controls the dynamics of axis convergence be important if the pseudocleavage furrow itself is dispensable for axis convergence.  
%Another interpretation of these two mechanisms is that the are part of a redundant system to align the AP axis.  If pseudocleavage fails, the cytoplasmic flows dependent mechanism can still drive axis convergence, albeit at a slower rate. Furthermore, incomplete axis convergence during the establishment phase can also be corrected later during maintenance phase \citep{mittasch2018non,gessele2020geometric}. This is suported by two previsous papers that show the pseudocleavage furrow may be dispensable for proper AP axis establishment \citep{rose1995pseudocleavage} and  other work that showed that the pseudocleavage furrow may play a role in the timing of AP axis establishment \citep{aras2018importance}. 
%How does \sout{then} this reconcile with the observation that the pseudocleavage furrow may be dispensable \citep{rose1995pseudocleavage,cowan2004asymmetric}?
% Nematics and Actin filament architecture
% Coupling between nematics, flows and geometry. Actin architecture important for axis convergence
%TO SI: Previous studies \citep{reymann2016cortical,aras2018importance} have investigated the mechanics of the pseudocleavage furrow only in the axisymmetric case - when the long axis and AP axis are aligned. Here, we expand on the description in Reymann et al. \citep{reymann2016cortical} to consider non-axisymmetric cases - when the AP axis is not along the long axis. In these cases, additional curvature sensitive stresses dependent on the coupling between nematics and cortical flows are revealed. These stresses arise as a result of the pseudocleavage furrow being away from its axisymmetric position. KEEP LAST SENTENCE? Our results show that this curvature sensitivity (arising as a result of considering actin nematics on a curved geometry in the non-axisymmetric case) is crucial for the shape sensitivity of axis convergence in \ac{ce} embryo. 
%Such curvature sensitive mechanism underlying axis convergence in \ac{ce} embryo raises the possibility that actin architecture and its modifications can impact axis convergence. Given the general importance of actomyosin cortex \citep{salbreux2012actin,gross2017review}, we suspect that interactions between actin nematics and cortical flows on curved surfaces that we have uncovered here are generally applicable to other biological systems.
%This broken symmetry, when coupled to the cortical flows, adds curvature sensitive stress to the cortex. Our results show that this curvature sensitivity is crucial for the shape sensitivity of axis convergence in \ac{ce} embryo. Such curvature sensitive mechanism underlying axis convergence in \ac{ce} embryo raises the possibility that actin architecture and its modifications can impact axis convergence. Given the general importance of actomyosin cortex \citep{salbreux2012actin,gross2017review}, we suspect that the interactions between actin nematics, cortical flows and cell or embryo shape that we have uncovered are general and likely to be seen in other systems as well (NOT PERFECT).
% Role of geometry and eggshell
% Changes in Eggshell
%Conclusion 
%Body axes in developing embryos often coincide with or converge along geometric features of the embryo or its surroundings \citep{schweizer2012geometry,etoc2016balance,zhang2019mouse,hiramatsu2013external,vianello2019mammalBioEngineer,mesnardMouse2004}. Our work in the \ac{ce} embryo shows that such geometric cues, in this case provided by the \sout{asymmetric} \color{green} ellipsoidal-like \color{black} shape of the embryo, can guide AP axis establishment via coupling between mechanical flows and embryo geometry, thereby driving the convergence of the AP axis onto the long axis. We propose that such a mechanism may be a general feature of body axis establishment. Note. observations in Mouse Drosophila chicken. Elucidating the general features of such 'geometric' guidance during body axes establishment convergence, and the role of mechanical flows in this process would be an interesting subject for future work.
%Sperm entry triggers pattern formation and direct body axes establishment in many species (REF C elegans, Xenopus, Ascidian, Algae, Plants, tunicate?, mouse, human?, Drosophila?). Our work shows how sperm-derived polarity cues can be modified by external geometric asymmetries in \ac{ce} embryo, allowing embryo shape to influence the mechanochemical feedback system (REF Peter) governing the AP axis establishment and robustly converge the AP axis onto the long axis. Given the coincidence and convergence of body axes with geometrical features in other systems, even those not dependent on sperm entry (Mouse, Min, Xenopus?, Human?, Ascidian?), such geometric guidance of body axes establishment could be a general phenomenon. Future work might be aimed at elucidating how mechanical flows in the developing embryo could couple to geometric features such as curvature (REF Nature Cell Monolayer Curvature) and shape to enable such geometric guidance of body axes establishment. 
%Our work reveals how a self-organized mechanochemical feedback system can be guided in their patterning process by external cues. While previous studies have made similar observations -- centrosome guided formation of PAR domains in \ac{ce} \citep{gross2019guiding}, adhesion pattern guided gastrulation in \textit{Tribolium} \citep{munster2018integrin} -- here we show that geometrical features such as embryo shape can provide such cues. Via the pseudocleavage furrow-mediated coupling between embryo geometry and mechanics of an active nematic cortex, the mechanochemical feedback system that defines the AP axis is guided such that the AP axis converges onto the geometric long axis of the embryo. Our study shows that such interplay between embryo geometry, polarization and actomyosin mechanics enables robust re-orientation of the AP axis along the long axis of the embryo, irrespective of where the AP axis establishment is first initiated.
%Coincidence and convergence of body axes with geometrical features has been observed in various systems \citep{hiramatsu2013external,vianello2019mammalBioEngineer,mesnardMouse2004} (REF Min system). Using insights from our work here, future work might be aimed at elucidating how actomyosin cortex, or mechanical forces in general, can enable self-organizing pattern formation systems that direct body axes establishment to follow geometric cues provided by the embryo and its surroundings (in \ac{ce} embryo, the eggshell). Understanding this coupling between geometry, mechanics and pattern formation could provide useful insights in how embryos select body axes orientation in a robust manner during development.
%Our work reveals how a self-organized mechanochemical feeback systems can be guided in their patterning process by external cues. While previous studies have made similar observations -- centrosome guided formation of PAR domains in \ac{ce} \citep{gross2019guiding}, adhesion pattern guided gastrulation in \textit{Tribolium} \citep{munster2018integrin} -- here we show that geometrical features such as embryo shape can provide such cues. %Our work with rounder embryos (\textit{ima-3} RNAi) reinforces this conclusion, indicating that axis convergence is guided by the asymmetries in the shape of the embryos. In the context of \ac{ce} embryo, the eggshell thus provides the proper asymmetry in shape needed for axis convergence \citep{johnston2006eggshell,aras2018importance}. 
%JUST RESTATE AXIS CONVERGENCE. IWe conclude that the coupling of geometric cues to mechanochemical self-organization may I DONT KNOW ..Convergence of fundamental axis. \citep{hiramatsu2013external,vianello2019mammalBioEngineer,mesnardMouse2004} (REF min system bacteria) COMBINE WOTH NEXT PARAGRAPH
%somewhere above> We describe a mehanism where Mechnaical stesses geenrated in the actomyosin cortex generate torques with respect to the surrpondng and drive axzis reorientation. Future work might be aimed to reeal if other processes of axis sestablishment or reorientation, such as the axis of the mitotic spindle in Hertwigs rule...or other example....are driven by active stresses geerntaied by actomsoin ...contribute...