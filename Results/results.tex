In this chapter, we describe the results from the various experiments conducted to investigate the mechanism of \ac{ap} axis alignment, along with their comparisons to the numerical simulations of the theoretical model of \ac{ap} axis alignment described in \autoref{ch:ActiveMatter}. This chapter follows \citep{} closely, with experiments performed by me, P. Gross and M. Kramer; and numerical simulations by M. Nestler. 

\section{Characterising \acs{ap} axis alignment in unperturbed embryos} \label{sec:apAxisAlignCharacteriseWT}
We first set out to characterise the \ac{ap} axis alignment process in unperturbed embryos. \enquote{Unperturbed} here refers to no genetic perturbations, such as no \ac{rnai} or mutations, apart from the addition of fluorescent tags. To this end, we undertook time-lapse microscopy of embryos from the SWG070 strain, which is labelled with \flurophoreLabel{\ac{nmy2}}{\ac{gfp}} and \flurophoreLabel{phDomain}{mCherry}. In these embryos, the male pronucleus can be observed as a dark circle in the cytoplasm, in the \flurophoreLabel{\ac{nmy2}}{\ac{gfp}} fluorescent channel -- as cytoplasmic myosin is excluded from the proncleus. The posterior domain as the depletion of \ac{nmy2} on the cortex near the male pronucleus. We characterise the \ac{ap} axis alignment process by tracking the position of the male pronucleus as it undergoes posteriorisation. We refer the reader to \autoref{sec:imageAnalysis} for details on the image analysis methods used to track the male pronucleus. 

We quantify two aspects of the posteriorisation of the male pronucleus -- its \enquote{angular position} and \enquote{posteriorisation velocity} (see \autoref{sec:imageAnalysis}). Angular position refers to the angle made between the long axis and the line connecting the center of the male pronucleus to the center of the embryo. Posteriorisation velocity is the component of the velocity of the male pronucleus that is parallel to the cortex (at the given angular position). Negative posteriorisation velocities indicate movement in direction of decreasing angular positions and thus towards the posterior end, positive posteriorisation velocity in direction of increasing angular positions and thus away from the posterior end. We also denote the end of posteriorisation as T = \SI{0}{\second}, and synchronise all movies to this time-point. 

Plotting the angular position as a function of time, we find that the angular positions generally decrease towards \SI{0}{\unitAngle} as time reaches closer to end of posteriorisation (T = \SI{0}{\second}). In embryos where the \ac{ap} axis is mis-aligned -- that is, with initial angular position greater than \SI{5}{\unitAngle} (\num{33} out of \num{57} embryos) -- the \ac{ap} axis re-aligns back towards the long axis. We thus confirm the observation made in \cite{goldstein1996specification} -- the \ac{ap} axis does align itself towards the long axis of the embryo, evidenced by the posteriorisation of the male pronucleus.

Plotting the posteriorisation velocity as a function of angular position (see \autoref{sec:statAnalysis} for details on binning), we find that the male pronucleus is, on average, moving towards the posterior end -- with higher speed at higher angular positions. This can also be observed as increasing magnitude of slope of the angular position vs time plots for higher angular positions. We thus conclude that the rate of \ac{ap} axis alignment is faster at higher angular positions: the male pronucleus moves faster towards the posterior end the further away from the posterior end it is.

We also measure the cortical flows in the unperturbed embryos -- see \autoref{sec:imageAnalysis} for methods. We observe an average cortical speed of \SI{4.12 +- 0.59}{\unitCrtxVel}. We also bin the observed cortical flows using angular positions of the male pronucleus -- see \autoref{sec:statAnalysis}. We find that the point where the cortical flows change sign correlates with the angular position, as expected from the mechanism of \ac{ap} axis establishment. These observed cortical flows are later used by the model of \ac{ap} axis alignment described in \autoref{ch:ActiveMatter} for calibration, to generate theoretical values of posteriorisation velocity as a function of angular positions -- see \autoref{subsec:expVsTheoryPcFurrow}.

\section{Cortical flows are required for \acs{ap} axis alignment}\label{sec:corticalFlowsRoleMlc4}
As discussed in \autoref{sec:ApAxisEstablishment}, cortical flows play an important role in proper \ac{ap} axis establishment. Do cortical flows also play a role in \ac{ap} axis alignment? To answer this question, we sought to impair cortical flows, and observe if posteriorisation of the male pronucleus is affected. We perform a \ac{rnai} of \geneExp{mlc-4} on worms of SWG070 strain, for a feeding time of \SI{24}{\unitRNAiTime}, to reduce cortical flow velocity (see \autoref{sec:rnaiMethods} for details on \ac{rnai}). MLC-4 is a conserved regulatory light chain present in \ac{nmy2}, and is required for the \ac{nmy2} myosin motor to function \citep{shelton1999nonmuscle}. We find that cortical flows are indeed reduced in \geneExp{mlc-4} \ac{rnai} embryos -- we observe an average cortical flow speed of \SI{1.45 +- 0.30}{\unitCrtxVel} in \geneExp{mlc-4} \ac{rnai} embryos compared to \SI{4.12 +- 0.59}{\unitCrtxVel} in unperturbed control embryos.

In \geneExp{mlc-4} \ac{rnai} embryos, the male pronucleus was manually tracked instead of being tracked using the image analysis pipeline described in \autoref{sec:imageAnalysis}. We find that the posteriorisation of the male pronucleus is suppressed in these \ac{rnai} embryos: from \num{18} out of \num{30} \ac{rnai} embryos in which the male pronucleus has an initial angular position greater than \SI{5}{\unitAngle}, \num{10} fail to posteriorize. We also observe almost no change in angular position of the male pronucleus in \ac{rnai} embryos over time, as well as very small posteriorisation velocities over different angular positions compared to those observed in unperturbed embryos. We thus conclude that cortical flows are essential for posteriorisation of the male pronucleus, and thus \ac{ap} axis alignment.

\section{Role of Pseudocleavage furrow in \acs{ap} axis alignment}\label{sec:PcFurrowRole}
In this section, we aim to understand the role of the pseudocleavage furrow in the \ac{ap} axis alignment process. In effect, we evaluate to which degree the two mechanisms described in \autoref{ch:APAxisIntro} and \autoref{ch:ActiveMatter} -- the cytoplasmic flow-dependent mechanism and the pseudocleavage furrow-dependent mechanism -- are important for the \ac{ap} axis alignment process. 

\subsection{Removing Pseudocleavage furrow via \acs{rnai}}\label{subsec:Nop1AndNop1Mel11}
As stated, we want to understand if the pseudocleavage furrow-dependent mechanism plays a significant role in \ac{ap} axis alignment. To do so, we quantify the posteriorisation of the male pronucleus in embryos lacking a pseudocleavage furrow. To generate such embryos, we perform a \ac{rnai} of \geneExp{nop-1} on worms of SWG070 strain, for a feeding time of \SI{24}{\unitRNAiTime} (see \autoref{sec:rnaiMethods} for details on \ac{rnai}). NOP-1 modulates activity of the small GTPase RHO-1, which is a major regulator of the activity of the actomyosin cortex in the \ac{ce} embryo \citep{tse2012nop1}. Embryos generated by worms which are mutant for NOP-1 (that is, possess a non-functional form of NOP-1) lack a pseudocleavage furrow \citep{rose1995pseudocleavage}. We find that the \ac{rnai} does indeed generate pseudocleavage furrow-deficient embryos. We also find that \ac{ap} axis alignment is suppressed in the \geneExp{nop-1} \ac{rnai} embryos: these embryos exhibit slower posteriorization velocity compared to unperturbed embryos at different angular positions, with angular positions slowly decaying towards \SI{0}{\unitAngle}. However, we also observe that average cortical flow speed is reduced in these \ac{rnai} embryos: \SI{4.12 +- 0.59}{\unitCrtxVel} in \ac{rnai} embryos compared to \SI{4.12 +- 0.59}{\unitCrtxVel} in unperturbed embryos.

Similar features are observed in the pseudocleavage-furrow deficient embryos generated using worms which are mutant for NOP-1. Using embryos from the SWG228 strain -- which is mutant for NOP-1 and is labelled with \flurophoreLabel{\ac{nmy2}}{\ac{gfp}} -- we observe slower posteriorisation velocity and suppressed changes in angular positions, along with slower cortical flows (\SI{4.12 +- 0.59}{\unitCrtxVel} in \ac{rnai} embryos compared to \SI{4.12 +- 0.59}{\unitCrtxVel} in unperturbed embryos).

Since we have already shown that cortical flows are required for \ac{ap} axis alignment, we sought to generate pseudocleavage furrow-deficient embryos with cortical flows comparable to those observed in unperturbed embryos (which have a pseudocleavage furrow). To generate such embryos, we perform a double \ac{rnai} of \geneExp{nop-1} and \geneExp{mel-11} on worms of SWG070 strain, for a feeding time of \SI{24}{\unitRNAiTime} (see \autoref{sec:rnaiMethods} for details on double \ac{rnai}). MEL-11 is a myosin phosphatase \citep{piekny2002rho} that suppresses the activity of myosin in the cortex \citep{najafabadi2022orchestrating}. We find that embryos generated using this double \ac{rnai} method lack a pseudocleavage furrow. Additionally, these double \ac{rnai} embryos have an average cortical flow speed of \SI{3.34 +- 0.52}{\unitCrtxVel}, comparable to those observed in unperturbed embryos (\SI{4.12 +- 0.59}{\unitCrtxVel}). We also find that the posteriorisation of the male pronucleus is reduced in these double \ac{rnai} embryos: evidenced by a slow change in angular position and slower posteriorisation velocities compared to unperturbed embryos. Furthermore, the posteriorisation of the male pronucleus in these double \geneExp{nop-1; mel-11} \ac{rnai} embryos is comparable to that observed in the single \geneExp{nop-1} \ac{rnai} embryos -- the change in cortical flow speed only marginally effected the posteriorisation of the male pronucleus. However, the male pronucleus in these pseudocleavage-deficient does posteriorise, albeit slowly compared to unperturbed embryos.

Altogether, these experimental results lead us to conclude that the pseudocleavage furrow plays a role in the proper alignment of the \ac{ap} axis at the rate observed in the unperturbed embryos, but is not essential for \ac{ap} axis alignment. Embryos deficient in the pseudocleavage furrow can still exhibit \ac{ap} axis alignment, albeit at a slower rate.

%Motivation: What role does pseudocleavage furrow play in \ac{ap} axis alignment?
%Question: Is \ac{ap} axis alignment reduced by removing pseudocleavage furrow?
%Experiment 1: Remove PC using \textit{nop-1} \ac{rnai}. Refer to RNAi methods
%Observations 1: Cortical flows are reduced - compare to WT here (and refer to mlc-4). Posteriorization velocities are decreased - compare to WT %(refer to figure). Angular position decay slowly to zero. 
%Add. Observation 1: Reduced nematic order in nop-1 (from reymann et al).
%Requirement: Need to get cortical flows close to WT atleast
%Experiment 2: Remove PC using \textit{nop-1} \ac{rnai}, using \textit{mel-11} RNAi to boost cortical flows. Refer to RNAi methods
%Observations 2: Cortical flows are slightly reduced - compare to WT here (and refer to nop-1). Posteriorization velocities are decreased - compare to WT and nop-1 (refer to figure). Angular position decay less slowly to zero. Histograms
%Conclusion: Removing PC reduces \ac{ap} axis alignment rate.

\subsection{Comparing numerical simulations to experimental results}\label{subsec:expVsTheoryPcFurrow}
\subsubsection{Full model accounts for \acs{ap} axis alignment in unperturbed controls}\label{subsubsec:fullModelForWT}
To further probe the role of the pseudocleavage furrow in \ac{ap} axis alignment, we turn to the theoretical model described in \autoref{ch:ActiveMatter}. As described there, the theoretical model considers two possible models of \ac{ap} axis alignment: cytoplasmic flow-dependent mechanism, and the pseudocleavage furrow-dependent mechanism. We aim here to evaluate the contributions of the two mechanisms to the posteriorisation of the male pronucleus in the unperturbed embryos. As these embryos exhibit a pseudocleavage furrow, we consider both mechanisms -- that is, we use the full model. Note that we use the average axes lengths of the unperturbed embryos as the axes length of the ellipsoid used in the theoretical model: \longAxisLength = \SI{28.7}{\unitLength} (semi-major axis), \shortAxisLength = \SI{16.4}{\unitLength} (semi-minor axes) -- see \autoref{sec:imageAnalysis} for methods.

To calibrate our model (see \autoref{ch:ActiveMatter}), we use the experimentally measured cortical flows in unperturbed embryos (see \autoref{subsec:corticalFlows}) for methodology). Specifically, we vary the model parameters -- hydrodynamic length \hydrodynamicLength, active force relaxation \activeRelaxLength and nematic stress relaxation \nematicLength -- until the cortical flows calculated by numerical simulations match the cortical flows observed in the experiments. This is done for a range of angular positions of the male pronucleus, using the average cortical flows observed for angular position bin (with bin width of \SI{3}{\unitAngle} -- see \autoref{sec:statAnalysis} for details on methodology). As the number of movies that exhibit angular positions beyond \SI{21}{\unitAngle} drops below, we only consider angular positions in the range \SIrange{0}{21}{\unitAngle} for calibration of the theoretical model. We find that, after the calibration, the calculated cortical flows match the experimentally observed cortical flows for the following set of model parameters: \hydrodynamicLength = \SI{10}{\unitLength}, \activeRelaxLength = \SI{11.5}{\square\unitLength\per\second}, \nematicLength = \SI{152.5}{\square\unitLength\per\second}. 

In the theoretical model, bulk cytoplasmic flows are determined uniquely from the calculated cortical flows via the no-slip boundary condition (see \autoref{ch:ActiveMatter}). We compare these calculated cytoplasmic flows -- using the calibrated model parameters -- to those observed in the unperturbed embryos (see \autoref{subsec:cytoFlows} for methodology), and find that the calculated cytoplasmic flows show good agreement with experimental cytoplasmic flows over a range of angular positions of the male pronucleus. From these comparisons, we conclude that the theoretical model can faithfully recapitulate the experimental cortical and cytoplasmic flows, for the selected set of model parameters.

We now compare the experimentally observed posteriorisation of the male pronucleus with that calculated in the numerical simulations using the calibrated model. To calculate the posteriorisation velocity of the male pronucleus, the sole remaining model parameter -- drag coefficient \dragCoefficient reflecting direct interactions between the male pronucleus and the cortex (see \autoref{ch:ActiveMatter}) --  is needed. For the set of model parameters obtained via calibration (described above), we set \dragCoefficient = \num{0.61} to ensure that the calculated posteriorisation velocity best match the observed posteriorization velocity as a function of angular position of the male pronucleus. With these model parameters, we find that the calculated posteriorisation velocity agree with experimentally observed average posteriorisation velocity in unperturbed embryos, for angular position upto \SI{21}{\unitAngle}. For higher angular positions, average posteriorisation velocity observed in experiments is faster compared to calculated posteriorisation velocity for the same angular position. By integrating the calculated posteriorisation velocity (as a function of angular position), we obtain a calculated trajectory of the male pronucleus -- referring to the calculated angular position of the male pronucleus as a function of time to posteriorisation. We set the calculated trajectory to be at \SI{0}{\unitAngle} angular position at the end of posteriorisation. We find that the calculated trajectory agrees well with our experimentally observed trajectories of the male pronucleus. 

Additionally, we investigate the sensitivity of the model parameters selected here, with respect to the change in calculated posteriorisation velocity. Specifically, we separately set \hydrodynamicLength, \nematicLength, \activeRelaxLength and \dragCoefficient to $\pm$\num{50}\% of their fitted values, and calculate the posteriorisation velocity. We refer here to these instances of the model with one varied parameter as varied models. On comparison of the calculated posteriorisation velocity, we observe that the varied models still retain the same qualitative behaviour as that exhibited by the calibrated model. Posteriorisation velocity calculated by the varied models for angular positions upto \SI{21}{\unitAngle} remain comparable to the average posteriorisation velocity observed in experiments. Posteriorisation velocity calculated by the varied models for higher angular positions are slower than the average posteriorisation velocity observed in experiments. We also observe that largest variation in calculated posteriorisation velocity is due to variation in \nematicLength -- with larger values of \nematicLength leading to faster posteriorisation velocity. Note that \nematicLength is the model parameter that controls the contribution of the pseudocleavage-furrow dependent mechanism in the theoretical model, as described in \autoref{ch:ActiveMatter}.

Of note also is the hydrodynamic length \hydrodynamicLength, whose value have been measured in previous studies \citep{saha2016determining,mayer2010anisotropies}. We observe here a hydrodynamic length of \hydrodynamicLength = \SI{10}{\unitLength} for the unperturbed embryo, which is close to the previous measurements of the hydrodynamic length ($\sim$\SI{14}{\unitLength} \citep{saha2016determining,mayer2010anisotropies}). Additionally, our analysis above indicates that the calculated posteriorisation velocity is robust towards variation in \hydrodynamicLength. Thus, the calibrated hydrodynamic length used in our theoretical model is in agreement with the previously observed measurements of the hydrodynamic length of the cortex. 

Altogether, we conclude that the full model (with both mechanisms included), using the set of parameters selected here, can -- both qualitatively and quantitatively -- recapitulate the observed \ac{ap} axis alignment process in the unperturbed embryos, for angular positions upto \SI{21}{\unitAngle}. We will refer to the model evaluated with this set of model parameters as the unperturbed model.

\subsubsection{Eliminating role of pseudocleavage furrow-dependent mechanism in model broadly captures the slower \acs{ap} axis alignment in pseudocleavage furrow-deficient embryos}\label{subsubsec:cytoModelForNop1Mel11}
We next investigate if the model can recapitulate the observed posteriorisation of the male pronucleus in the pseudocleavage furrow-deficient embryos generated using \geneExp{nop-1; mel-11} \ac{rnai}. To mimic the experimental removal of the pseudocleavage furrow in the model, we fix \nematicLength -- the model parameter that controls the contribution of the pseudocleavage-furrow dependent mechanism in the model -- to \SI{0}{\square\unitLength\per\second}. Due to this, and since the cortical flows in these embryos are similar but not identical to those observed in unperturbed embryos, we calibrate the model again to ensure we capture the cortical flow velocity observed in the double \ac{rnai} embryos. Doing so yields the following model parameters: \hydrodynamicLength = \SI{11}{\unitLength}, \activeRelaxLength = \SI{7}{\square\unitLength\per\second}, \nematicLength = \SI{0}{\square\unitLength\per\second}. We retain the same value for the drag coefficient \dragCoefficient = \num{0.61} as determined for the unperturbed embryos. We will refer to the model evaluated with this set of model parameters as the pseudocleavage furrow-deficient model.

We compare the posteriorisation velocity calculated using the unperturbed model and the pseudocleavage-deficient model. We find that at each angular position, the pseudocleavage furrow-deficient model calculates slower posteriorisation velocity compared to the unperturbed model -- with the difference larger for higher angular positions. This is qualitatively similar to what was observed experimentally observed -- pseudocleavage furrow-deficient embryos exhibit slower average posteriorisation velocities compared to unperturbed embryos. Furthermore, we find that the fold change (ratio between the compared quantities) between the posteriorisation velocity calculated using the two models -- unperturbed and pseudocleavage-deficient -- is close to fold change between the experimentally observed average posteriorisation velocity in the unperturbed and pseudocleavage furrow-deficient embryos. However, on direct comparison between the calculated posteriorisation velocity from the pseudocleavage furrow-deficient embryos and experimentally observed average posteriorisation velocity in pseudocleavage furrow-deficient embryos, we find that the model calculates slower posteriorisation velocity for all angular positions considered here. Note that the calculated posteriorisation velocity is still comparable to the experimentally observed average posteriorisation velocity in pseudocleavage furrow-deficient embryos. This is further evidenced by comparing the male pronucleus trajectory calculated using the pseudocleavage furrow-deficient embryo with the experimentally observed male pronucleus trajectories. We calculate the male pronucleus trajectory for the pseudocleavage furrow-deficient model in the same fashion as done for the unperturbed model. We find that the calculated trajectory using the pseudocleavage furrow-deficient model agrees well with experimentally observed trajectories in the pseudocleavage-deficient embryos generated using \geneExp{nop-1; mel-11} \ac{rnai}.

We thus conclude that the posteriorisation of the male pronucleus calculated using the pseudocleavage furrow-deficient model -- where \nematicLength is set to \SI{0}{\square\unitLength\per\second} -- is comparable to the experimentally observed posteriorisation in the \geneExp{nop-1; mel-11} \ac{rnai} embryos. The unperturbed and pseudocleavage furrow-deficient models capture the fold change between the unperturbed and double \ac{rnai} embryos. However, the pseudocleavage furrow-deficient model predicts slightly slower posteriorisation than observed in the \geneExp{nop-1; mel-11} \ac{rnai} embryos.

%Note that pseudocleavage furrow is set to zero in model. Cortical flow calibration for nop-1/mel-11 embryos. Why do again calibration: because cortical flows are similar but not same. Show parameters. Comparison for posteriorization velocity and angular position with theory results - in that order. Conclusion: similar order, seems to capture behaviour.
\subsubsection{Suppressed pseudocleavage furrow-dependent mechanism better explains \acs{ap} axis alignment in pseudocleavage furrow-deficient embryos}\label{subsubsec:reducedPcModelForNop1Mel11}
We have reduced the expression of \geneExp{nop-1} (via \ac{rnai}) in order to generate pseudocleavage furrow-deficient embryos. Given that the pseudocleavage furrow-dependent mechanism arises due to the nematic nature of the cortex (see \autoref{ch:ActiveMatter}), we look into the effect \geneExp{nop-1} \ac{rnai} has on nematic order in the cortex. \ac{rnai} of \geneExp{nop-1} reduces the nematic order of the cortex, as measured in \cite{reymann2016cortical} by characterising the orientation of actin filaments. However, note that the cortex in these embryos still retains a weak nematic nature.

Motivated by this observation, we ask if recalibrating the model for the pseudocleavage furrow-deficient embryos generated using \geneExp{nop-1; mel-11} \ac{rnai} while allowing \nematicLength to be non-zero would help explain the discrepancy between the experimentally observed average posteriorisation velocity in the double \ac{rnai} embryos and calculated posteriorisation velocity from the pseudocleavage furrow-deficient embryos. We thus recalibrate the model using the cortical flows experimentally observed in pseudocleavage furrow-deficient embryos , and allowing \nematicLength to vary during the recalibration. We still retain the drag coefficient \dragCoefficient = \num{0.61} -- same as used for the unperturbed model. Doing so yields the following model parameters: \hydrodynamicLength = \SI{11}{\unitLength}, \activeRelaxLength = \SI{7}{\square\unitLength\per\second}, \nematicLength = \SI{25}{\square\unitLength\per\second}. We will refer to the model evaluated with this set of model parameters as the weak pseudocleavage furrow model. Note that the change in \nematicLength between the pseudocleavage furrow-deficient model and the weak pseudocleavage furrow model is much smaller ($\sim$\num{20}\%) compared to the difference between the pseudocleavage furrow-deficient model and the unperturbed model. 

We then compare the posteriorisation velocity calculated by the weak pseudocleavage furrow model with the experimentally observed average posteriorisation velocity in pseudocleavage furrow-deficient embryos (generated using \geneExp{nop-1; mel-11} \ac{rnai}). We find that the calculated posteriorisation velocity using the weak pseudocleavage furrow model better match the experimentally observed average posteriorisation velocity in the pseudocleavage furrow-deficient embryos than those calculated by the pseudocleavage furrow-deficient model. Note however that the difference between the posteriorisation velocities calculated using the pseudocleavage furrow-deficient model and the weak pseudocleavage furrow model is much smaller than between either and those calculated using the unperturbed model.

\subsubsection{Pseudocleavage furrow-dependent mechanism is the predominant mechanism for \acs{ap} axis alignment in unperturbed embryos}\label{subsubsec:pcFurrowDominatesConclude}
Using the models presented in the above section, we can now evaluate the importance of the pseudocleavage furrow-dependent mechanism to the \ac{ap} axis alignment process in unperturbed embryos. We make the following observations:
\begin{itemize}
    \item Posteriorisation velocity calculated by the unperturbed model is mostly sensitive to variations in \nematicLength, and is robust against variations in \hydrodynamicLength and \activeRelaxLength. We retain the same \dragCoefficient for all models.
    \item The model parameter with the largest variation between the three models -- unperturbed, pseudocleavage furrow-deficient, and weak pseudocleavage furrow -- is the \nematicLength. In particular, \nematicLength for the unperturbed model is much larger compared to those for the weak pseudocleavage furrow and the pseudocleavage furrow-deficient models. Other model parameters do not vary as much as the \nematicLength between models.
\end{itemize}
Together, these observations indicate that the large difference between the \nematicLength in the unperturbed model and the weak pseudocleavage furrow model explain most of the difference between the experimentally observed average posteriorisation velocity in the unperturbed embryos and the pseudocleavage furrow-deficient embryos (generated using \geneExp{nop-1; mel-11} \ac{rnai}). This similarly follows for the pseudocleavage furrow-deficient model. 

Note that \nematicLength is the model parameter that controls the contribution of the pseudocleavage furrow-dependent mechanism in the model (see \autoref{ch:ActiveMatter}). Also note that the posteriorisation velocity is much slower in the pseudocleavage furrow-deficient embryos compared to unperturbed embryos ($\sim$\num{80}\% slower). Combined with the observations above, we are led to conclude that the pseudocleavage furrow-dependent mechanism provides the major contribution to the posteriorisation velocity observed in the unperturbed embryos, and thus is the predominant mechanism responsible for the proper \ac{ap} axis alignment in the unperturbed embryos. Slow \ac{ap} axis alignment in the pseudocleavage-furrow deficient embryos indicates that the other mechanism -- the cytoplasmic flow-dependent mechanism -- plays a minor role in \ac{ap} axis alignment.

%Note that pseudocleavage furrow is allowed to vary. Cortical flow calibration for nop-1/mel-11 embryos. Comparison for posteriorization velocity and angular position with theory results - in that order. Show parameters. Conclusion: better match, furrow parameter still low compared to unperturbed.

\section{Role of embryo geometry in \acs{ap} axis alignment}\label{sec:GeometryRole}
In this section, we aim to understand the role of embryo geometry in the \ac{ap} axis alignment process. Here, we consider changes in the aspect ratio of the embryo, while still retaining an ellipsoidal geometry. Specifically, we ask how the rate of \ac{ap} axis alignment, measured as the posteriorisation velocity of the male pronucleus, differs between ellipsoidal embryos with different aspect ratios. We define aspect ratio of the embryo as the ratio of the lengths of the long axis 2\longAxisLength and the two equal short axes 2\shortAxisLength~ -- aspect ratio = \aspectRatio. 

\subsection{Rounder embryos show slower \acs{ap} axis alignment}\label{subsec:roundEmbryosIma3}
As stated, we want to understand if and how changes in embryo geometry affect the rate of \ac{ap} axis alignment -- measured here as the posteriorisation velocity of the male pronucleus. The specific change in embryo geometry we consider here is the reduction in aspect ratio of the embryo -- leading to embryos with a more spherical or \enquote{rounder} shape (hence referred to as rounder embryos). 

We first approach this question using numerical simulations of the theoretical model. We use the same model parameters as defined for the unperturbed model (as defined in \autoref{subsec:expVsTheoryPcFurrow}) -- thus, the model parameters chosen are: \hydrodynamicLength = \SI{10}{\unitLength}, \activeRelaxLength = \SI{11.5}{\square\unitLength\per\second}, \nematicLength = \SI{152.5}{\square\unitLength\per\second}, and \dragCoefficient = \num{0.61}. However, we now vary the aspect ratio of the ellipsoid used for the numerical simulation, while keeping the volume of the ellipsoid the same as that used for the unperturbed model (see \autoref{ch:ActiveMatter} for details). Specifically, we consider ellipsoids with the following aspect ratios: \aspectRatio = \num{1.10}, \num{1.25}, \num{1.35}, \num{1.48}, \num{1.58}, \num{1.78}. Note that \num{1.78} is the aspect ratio of the ellipsoid used for the unperturbed model -- corresponding to the average aspect ratio of unperturbed embryos (see below). We calculate the predicted posteriorisation velocity as a function of angular position of the male pronucleus, using each of the aspect ratios listed before and the model parameters set before. As for the unperturbed model, we consider only angular positions upto \SI{21}{\unitAngle}. We find that for each angular position, posteriorisation velocity generally grow faster as the aspect ratio rises -- with larger differences between posteriorisation velocities at higher angular positions. Thus, we conclude that the model predicts that the rate of \ac{ap} axis alignment is related to the aspect ratio of the ellipsoid, and therefore embryo geometry -- with rounder embryos exhibiting slower \ac{ap} axis alignment compared to more ellipsoidal ones. Note that such a prediction is consistent with the extreme case of a perfectly round (or spherical) embryo; since such a spherical embryo lacks any unique long axis, we must expect no \ac{ap} axis alignment in a spherical embryo.

We next verify this prediction experimentally. To do so, we generate rounder embryos by performing a \ac{rnai} of \geneExp{ima-3} on worms of SWG070 strain, for a feeding time of \SI{20}{\unitRNAiTime} (see \autoref{sec:rnaiMethods} for details on \ac{rnai}). IMA-3 is a member of the importin $\alpha$ family of nuclear transport factors \citep{geles2001germline,sonnichsen2005full}. It has been observed in previous studies that \geneExp{ima-3} \ac{rnai} generates rounder and smaller embryos \citep{sonnichsen2005full}. We verify this here by measuring the lengths of the semi-major \longAxisLength and semi-minor \shortAxisLength axes and the aspect ratio of \geneExp{ima-3} \ac{rnai} embryos and comparing them to those measured for the unperturbed embryos (see \autoref{subsec:nucleusTracking} for details). We find that the average axes lengths for the unperturbed embryos was \longAxisLength = \SI{28.7 +- 1.6}{\unitLength} and \shortAxisLength = \SI{16.2 +- 1.1}{\unitLength}, corresponding to an aspect ratio of \aspectRatio = \num{1.78 +- 0.20}. This is compared to the average axes lengths and aspect ratio measured for \geneExp{ima-3} \ac{rnai} embryos, for which we find \longAxisLength = \SI{20.9 +- 2.2}{\unitLength}, \shortAxisLength = \SI{13.8 +- 0.8}{\unitLength} and \aspectRatio = \num{1.48 +- 0.20}. We also note that the average volume of these embryos (calculated for each embryo as the volume of the ellipsoid with the same axes lengths as those measured for the embryo) is smaller in \geneExp{ima-3} \ac{rnai} embryos compared to unperturbed embryos -- indicating \geneExp{ima-3} \ac{rnai} embryos have a smaller volume compared to unperturbed embryos. Therefore, we note that \geneExp{ima-3} \ac{rnai} embryos are, on average, smaller and rounder compared to unperturbed embryos.

We also investigate if the model parameters used for the unperturbed model can be utilized for the \ac{rnai} embryos. Specifically, we compare the measured myosin concentrations in the cytosol and cortex of the \ac{rnai} embryos with those measured in the unperturbed embryos (see \autoref{subsec:nucleusTracking} for methods); and also compare the cortical flows measured in the \ac{rnai} embryos to those measured in the unperturbed embryos. We find that the myosin concentrations in the cytosol and cytoplasm are not significantly different between the \ac{rnai} embryos and the unperturbed embryos. We also find that the average cortical flow speed is not significantly different between the \ac{rnai} embryos and the unperturbed embryos: \SI{4.12 +- 0.59}{\unitCrtxVel} in \ac{rnai} embryos compared to \SI{4.12 +- 0.59}{\unitCrtxVel} in unperturbed embryos. These observations indicate that the cortex of the \ac{rnai} embryos does not differ from the cortex of the unperturbed embryos in any meaningful way. Thus, we conclude that the same model parameters used for the unperturbed model can also be used to calculate predicted posteriorisation velocity for the \geneExp{ima-3} \ac{rnai} embryos.

Having confirmed that the same model parameters as those used for comparison to the unperturbed embryos can be used for comparison to the \geneExp{ima-3} \ac{rnai} embryos, we next use the model to predict the posteriorisation velocity of the male pronucleus for the rounder \ac{rnai} embryos. For the model, we use an ellipsoid with aspect ratio \aspectRatio = \num{1.48}, and the same volume as that used for unperturbed embryos -- yielding semi-major axis of length \longAxisLength = \SI{25.8}{\unitLength} and semi-minor axes of length \shortAxisLength = \SI{17.4}{\unitLength}. Therefore, our model prediction ignores the difference in embryo size between the \ac{rnai} embryos and unperturbed embryos. We compare the posteriorisation velocity calculated using in the numerical simulations of this model to the observed average posteriorisation velocity in the \ac{rnai} embryos as a function of angular positions. We find that the predicted posteriorisation velocity agree with experimentally observed average posteriorisation velocity in \ac{rnai} embryos, for angular position upto \SI{24}{\unitAngle}. For higher angular positions, average posteriorisation velocity observed in experiments is faster compared to predicted posteriorisation velocity for the same angular position. By integrating the predicted posteriorisation velocity (as a function of angular position), we obtain a predicted trajectory of the male pronucleus -- referring to the predicted angular position of the male pronucleus as a function of time to posteriorisation. We set the predicted trajectory to be at \SI{0}{\unitAngle} angular position at the end of posteriorisation. We find that the predicted trajectory agrees well with our experimentally observed trajectories of the male pronucleus for the \ac{rnai} embryos.

We next compare the posteriorisation of the male pronucleus in these \ac{rnai} embryos to that observed in the unperturbed embryos. Comparing the average posteriorisation velocity as a function of angular positions, we observe that the rounder \ac{rnai} embryos exhibits slower average posteriorisation velocity compared to those measured in unperturbed embryos for all angular positions -- with larger differences for higher angular positions. We also observe this in the male pronucleus trajectories, with slower change in angular position as a function of time in \ac{rnai} embryos, compared to unperturbed embryos. Thus, we conclude that rounder \geneExp{ima-3} \ac{rnai} embryos exhibit slower rate of \ac{ap} axis alignment compared to unperturbed embryos -- matching the prediction of the theoretical model. Note also that while the posteriorisation velocity is slower in \geneExp{ima-3} \ac{rnai} embryos compared to those observed in unperturbed embryos, but is faster than those observed in the pseudocleavage furrow-deficient embryos generated by the \geneExp{nop-1; mel-11} \ac{rnai}.

Altogether, these observations lead us to conclude that rounder embryos do indeed show slower rate of \ac{ap} axis alignment -- as predicted by our theoretical model and confirmed experimentally. Since the diminished posteriorisation velocity observed in the rounder \geneExp{ima-3} \ac{rnai} embryos match the posteriorisation velocity predicted by the model using the same aspect ratio as the average aspect ratio observed in \geneExp{ima-3} \ac{rnai} embryos, our model captures this relation between embryo geometry and \ac{ap} axis alignment in a quantitative manner. Additionally, note that the prediction for the \geneExp{ima-3} \ac{rnai} embryos only takes into account the change in aspect ratio of the embryos, yet makes a prediction that matches the experimental observations. Thus, we conclude that the primary feature of embryo geometry that influences \ac{ap} axis alignment is the aspect ratio of the embryo.

%Motivation: How does changing embryo geometry change \ac{ap} axis alignment?
%Question: What happens for rounder embryos - less prominent long axis.
%Prediction: Less prominent long axis implies less prominent alignment. Numerical simulations on rounder embryos with same volumes indicate this is true.
%Experiment: Rounder embryos using ima-3 RNAi. Refer to methods.
%Observations 1: ima-3 embryos are rounder (long axis, short axis, aspect ratio). Same myosin concentrations and cortical flows as WT. Volumes are reduced - 70\%.
%Observations 2: Posteriorization velocities are decreased - compare to WT and nop-1/mel-11 (refer to figure). Angular position does decay slightly slower to zero. Histograms
%Conclusion: Rounder embryos show slower \ac{ap} axis alignment

\subsection{Relation between embryo geometry and \acs{ap} axis alignment}\label{subsec:postVelVsAspectRatio}

\subsubsection{Exploring the relation between aspect ratio and posteriorisation velocity}\label{subsubsec:postVelVsAspectRatioExpAndModel}
Our theoretical model demonstrates a relation between aspect ratio of the embryo and the rate of \ac{ap} axis alignment (measured as posteriorisation velocity of the male pronucleus): slower rate of \ac{ap} axis alignment (that is, slower posteriorisation velocity at different angular positions) for smaller aspect ratios (that is, rounder embryos). We next sought to experimentally evaluate this relation directly. Here, instead of considering the posteriorisation velocity of the male pronucleus as a function of angular position for a given average aspect ratio (as we have done until now), we now consider the posteriorization velocity as a function of aspect ratio, averaged over a range of angular positions. Experimental variation in the aspect ratios of both unperturbed (\numrange{1.6}{2.1}) and \geneExp{ima-3} \ac{rnai} embryos (\numrange{1.1}{1.7}) enables us to evaluate this relation over a broad range of aspect ratios (\numrange{1.1}{2.1}). Since the difference in volume between the unperturbed embryos and \ac{rnai} embryos does not appear to be important for \ac{ap} axis alignment, and that cortical flows and myosin concentrations are not significantly different between unperturbed embryos and \geneExp{ima-3} \ac{rnai} embryos; we combine the two sets of embryos into a single set which we consider here. Thus, the set of embryos considered in this subsection contains both the unperturbed embryos and \ac{rnai} embryos, with no distinction made between the two. We will refer to this set of embryos as the combined set of embryos.

We next proceed to obtain the experimentally observed relation between the posteriorisation velocity of the male pronucleus and aspect ratio of the embryo from the combined set of embryos. To do so, we bin the combined set using aspect ratio, with bin width of \num{0.2}. For each bin, we average over the observed posteriorisation velocity for all angular positions between \SIrange{3}{15}{\unitAngle}. This gives us the average posteriorisation velocity (along with \num{95}\% confidence intervals calculated using two sided t-test) observed for each aspect ratio bin -- thus yielding the experimentally observed relation between the two.

From the numerical simulations done using ellipsoids with different aspect ratios, we can also obtain a relation between the posteriorisation velocity and aspect ratio as predicted by the theoretical model. For each aspect ratio used before, we average the posteriorisation velocity for all angular positions between \SIrange{3}{15}{\unitAngle}. For aspect ratios in-between, we use linear interpolation. This yields us the relation between the posteriorisation velocity and aspect ratio, as calculated using the theoretical model. Comparing to the experimentally observed relation, we find that the theoretical relation matches the experimental one well over a range of aspect ratios from \numrange{1.1}{1.7}.

\subsubsection{Effective model of a contractile ring on an ellipsoid captures the relation between embryo geometry and \acs{ap} axis alignment}\label{subsubsec:effectiveModel}

How does this relation between the embryo geometry (characterised by aspect ratio) and \ac{ap} axis alignment (characterised by posteriorisation velocity) arise? As shown in \autoref{subsec:expVsTheoryPcFurrow}, the predominant mechanism for \ac{ap} axis alignment is the pseudocleavage furrow-dependent mechanism. The pseudocleavage furrow may effectively be considered as a contractile ring, rotating about the equator of an ellipsoid (representing the embryo) during \ac{ap} axis alignment. We here ask if this effective model of a contractile ring on an ellipsoid enough to capture the relation between aspect ratio of the embryo and posteriorisation velocity of the male pronucleus, as observed in the combined set of embryos.

In this effective model, we consider an fixed axi-symmetric ellipsoid of axis lengths [\longAxisLength,\shortAxisLength,\shortAxisLength], with \longAxisLength being the length of the semi-major axis and \shortAxisLength being the length of the two equal semi-minor axes. The semi-major (or long) axis defines the x-axis, while the two semi-major (or short) axes define the y- and z- axes, respectively. On this ellipsoid, we consider a contractile ring which represents the pseudocleavage furrow. The center of the ring and the center of the ellipsoid are coincident (at the origin of the coordinate system). This ring rotates about the z-axis, representing the rotation of the pseudocleavage furrow during \ac{ap} axis alignment. Thus, the contractile ring has only a sole degree of freedom: the angle $\alpha$ made between the normal to the plane of the contractile ring and the x-axis. Since we are only interested in angular positions from \SIrange{3}{15}{\unitAngle}, we only consider small angle $\alpha$ here.

We make the following assumptions in our effective model, with regards to the dynamics of the contractile ring:
\begin{itemize}
    \item The ring has a constant line tension $T$ - independent of both the orientation of the ring ($\alpha$) and the shape of the ellipsoid (\longAxisLength, \shortAxisLength). Here, $T$ has units of force.
    \item The frictional torque acting on the ring is controlled by a frictional coefficient $\gamma$, and is given by $\gamma\cdot\dot{\alpha}$
    \item We do not consider any inertial terms: Torque generated by the ring is perfectly balanced by the torque generated by friction.
\end{itemize}                                                                             
Under the above assumptions, we want to calculate $\dot{\alpha}$ as a function of \longAxisLength, \shortAxisLength and $\alpha$ -- for small angles $\alpha$. Note that \longAxisLength and \shortAxisLength are free to vary - we do not assume that our ellipsoid is close to a sphere.

\paragraph{Motion of contractile ring}
We first consider the mechanical energy stored in the contractile ring, which can be given by:
\begin{equation} \label{eq:energyDef}
    E = TC(\alpha)
\end{equation}
where $C(\alpha)$ is the total circumference of the ring.

To calculate the circumference of the ring for any given $\alpha$, we first write a description of the ring as an intersection of the ellipsoid with the plane in which the ring resides. This plane makes an angle of $\alpha$ with the yz plane, and passes through the origin. Thus, it can be described by the equation: $y = -x\cot\alpha$. The ellipsoid itself can be described as $\frac{x^2}{a^2} + \frac{y^2 + z^2}{b^2} = 1$. On taking the intersection of the ring plane with the ellipsoid, we get the equation describing the ring:
\begin{align*}
    \textrm{Plane: }&y = -x\cot\alpha \\
    \textrm{Ellipsoid: }&\frac{x^2}{a^2} + \frac{y^2 + z^2}{b^2} = 1 \\
    \textrm{Intersection (Ellipse): }&\frac{x^2}{a^2} + \frac{x^2\cot^2\alpha + z^2}{b^2} = 1\\
    &\implies x^2\left(\frac{a^2\cot^2\alpha + b^2}{a^2b^2}\right) + \frac{z^2}{b^2} = 1; \quad y = -x\cot\alpha
\end{align*}
Let us define $A = \frac{ab}{\sqrt{a^2\cot^2\alpha + b^2}}$. Then, the equation in the intersection part above can be written as: $\frac{x^2}{A^2} + \frac{z^2}{b^2} = 1$ - a form similar to the canonical form for an ellipse. Calling the parametric angle for this ellipse $\phi$, we then can write the position vector describing the ring $\Vec{r}$ as:
\begin{equation*}
    \Vec{r} = \left(A\cos\phi,-A\cot\alpha\cos\phi,b\sin\phi\right); \quad A = \frac{ab}{\sqrt{a^2\cot^2\alpha + b^2}}
\end{equation*}
where $\phi \in [-\pi,\pi)$.

Note that the ring is an ellipse in its plane - since the intersection of an ellipsoid and a plane must be an ellipse. This can also be seen by the form by $\norm{\Vec{r}}^2$:
\begin{equation*}
    \norm{\Vec{r}}^2 = A^2(1 + \cot^2\alpha)\cos^2\phi + b^2\sin^2\phi
\end{equation*}

We can therefore calculate semi-major $a_{ring}$ and semi-minor $b_{ring}$ axes of the ellipse formed by the ring. We directly read them off the expression of $\norm{\Vec{r}}^2$ :
\begin{align*}
    a_{ring} &= A\sqrt{1 + \cot^2\alpha} = \frac{ab}{\sqrt{a^2\cot^2\alpha + b^2}}\sqrt{1 + \cot^2\alpha^2} = \frac{ab}{\sqrt{a^2\cos^2\alpha + b^2\sin^2\alpha}}\\
    b_{ring} &= b \\
    e_{ring} &= \sqrt{1 - \left(\frac{b_{ring}}{a_{ring}}\right)^2} = \sqrt{1 - \frac{a^2\cos^2\alpha + b^2\sin^2\alpha}{a^2}} = \sqrt{\sin^2\alpha - \frac{b^2}{a^2}\sin^2\alpha} = \sin\alpha\sqrt{1 - \frac{b^2}{a^2}}
\end{align*}
where $e_{ring}$ is the eccentricity of the ellipse formed by the ring. Note that since the ring rotates about the yz plane (i.e. the plane with the two short axes), the short axis of the ring is just $b$.

The circumference of the ring is then given by the circumference of this ellipse. The formula for the circumference of an ellipse is:
\begin{equation} \label{eq:circum}
    C(\alpha) = 4a_{ring}\int_0^{\frac{\pi}{2}} \sqrt{1 - e_{ring}^2\sin^2\phi} \dd{\phi}
\end{equation}

\paragraph{Torque generated by contractile ring}
Using \eqref{eq:energyDef} and \eqref{eq:circum}, we can write down the energy as:
\begin{align*}
    E &= TC(\alpha) = 4Ta_{ring}\int_0^{\frac{\pi}{2}} \sqrt{1 - e_{ring}^2\sin^2\phi} \dd{\phi}\\
    &= 4T\frac{ab}{\sqrt{a^2\cos^2\alpha + b^2\sin^2\alpha}}\int_0^{\frac{\pi}{2}} \sqrt{1 - \left(1 - \frac{b^2}{a^2}\right)\sin^2\alpha\sin^2\phi} \dd{\phi}
\end{align*}

The torque can then be expressed as:
\begin{align*}
    \tau &= -\dv{E}{\alpha} = -4T\dv{\alpha}\left(\frac{ab}{\sqrt{a^2\cos^2\alpha + b^2\sin^2\alpha}}\int_0^{\frac{\pi}{2}} \sqrt{1 - \left(1 - \frac{b^2}{a^2}\right)\sin^2\alpha\sin^2\phi_{ring}} \dd{\phi_{ring}}\right)\\
    &= -4T\left[\frac{ab(a^2 - b^2)\cos\alpha\sin\alpha}{\left(a^2\cos^2\alpha + b^2\sin^2\alpha\right)^{\flatfrac{3}{2}}}\int_0^{\frac{\pi}{2}} \sqrt{1 - \left(1 - \frac{b^2}{a^2}\right)\sin^2\alpha\sin^2\phi_{ring}} \dd{\phi_{ring}}\right.\\
    &\left.\quad + \frac{ab}{\sqrt{a^2\cos^2\alpha + b^2\sin^2\alpha}}\int_0^{\frac{\pi}{2}} \frac{-\left(1 - \frac{b^2}{a^2}\right)\sin\alpha\cos\alpha\sin^2\phi_{ring}}{\sqrt{1 - \left(1 - \frac{b^2}{a^2}\right)\sin^2\alpha\sin^2\phi_{ring}}} \dd{\phi_{ring}}\right]\\
    &= -4T\frac{ab(a^2 - b^2)\sin\alpha\cos\alpha}{\sqrt{a^2\cos^2\alpha + b^2\sin^2\alpha}}\\
    &\times\int_0^{\frac{\pi}{2}}\left[\frac{1 - \left(1 - \frac{b^2}{a^2}\right)\sin^2\alpha\sin^2\phi_{ring}}{a^2\cos^2\alpha + b^2\sin^2\alpha}-\frac{\sin^2\phi_{ring}}{a^2}\right]\frac{1}{\sqrt{1 - \left(1 - \frac{b^2}{a^2}\right)\sin^2\alpha\sin^2\phi_{ring}}}\dd{\phi_{ring}}\\
    &= -4T\frac{ab(a^2 - b^2)\sin\alpha\cos\alpha}{a^2\sqrt{a^2\cos^2\alpha + b^2\sin^2\alpha}}\\
    &\times\int_0^{\frac{\pi}{2}}\left[\frac{a^2 - \left(a^2 - b^2\right)\sin^2\alpha\sin^2\phi_{ring}}{a^2\cos^2\alpha + b^2\sin^2\alpha}-\sin^2\phi_{ring}\right]\frac{1}{\sqrt{1 - \left(1 - \frac{b^2}{a^2}\right)\sin^2\alpha\sin^2\phi_{ring}}}\dd{\phi_{ring}}\\
    &= -4T\frac{ab(a^2 - b^2)\sin\alpha\cos\alpha}{a^2\sqrt{a^2\cos^2\alpha + b^2\sin^2\alpha}}\\
    &\times\int_0^{\frac{\pi}{2}}a^2\left[\frac{1 - \sin^2\phi_{ring}}{a^2\cos^2\alpha + b^2\sin^2\alpha}\right]\frac{1}{\sqrt{1 - \left(1 - \frac{b^2}{a^2}\right)\sin^2\alpha\sin^2\phi_{ring}}}\dd{\phi_{ring}}\\
    \implies \tau &= -4T\frac{ab(a^2 - b^2)\sin\alpha\cos\alpha}{\left(a^2\cos^2\alpha + b^2\sin^2\alpha\right)^{\flatfrac{3}{2}}}\int_0^{\frac{\pi}{2}}\frac{\cos^2\phi_{ring}}{\sqrt{1 - \left(1 - \frac{b^2}{a^2}\right)\sin^2\alpha\sin^2\phi_{ring}}}\dd{\phi_{ring}}
\end{align*}

Using the Taylor expansion around $\alpha = 0$, we can expand $\tau$ to linear order in $\alpha$ as $\tau = \left.\tau\right|_{\alpha=0} + \left.\dv{\tau}{\alpha}\right|_{\alpha=0}\alpha + \mathcal{O}(\alpha^2)$. Note that on $\alpha = 0$, the torque also goes to zero. To obtain the linear order expansion, we calculate the derivative of the torque at $\alpha = 0$.
\begin{align*}
    \left.\dv{\tau}{\alpha}\right|_{\alpha=0} &= -4T\left.\dv{\alpha}\left[\frac{ab(a^2 - b^2)\sin\alpha\cos\alpha}{\left(a^2\cos^2\alpha + b^2\sin^2\alpha\right)^{\flatfrac{3}{2}}}\int_0^{\frac{\pi}{2}}\frac{\cos^2\phi_{ring}}{\sqrt{1 - \left(1 - \frac{b^2}{a^2}\right)\sin^2\alpha\sin^2\phi_{ring}}}\dd{\phi_{ring}}\right]\right|_{\alpha = 0} \\
    &= -4T \left[\dv{\alpha}\left(\frac{ab(a^2 - b^2)\sin\alpha\cos\alpha}{\left(a^2\cos^2\alpha + b^2\sin^2\alpha\right)^{\flatfrac{3}{2}}}\right)_{\alpha=0}\int_0^{\frac{\pi}{2}}\frac{\cos^2\phi_{ring}}{\sqrt{1 - \left(1 - \frac{b^2}{a^2}\right)(0)\sin^2\phi_{ring}}}\dd{\phi_{ring}}\right.\\
    &\quad + \left.\frac{ab(a^2 - b^2)*0*1}{\left(a^2(1) + b^2(0)\right)^{\flatfrac{3}{2}}}\dv{\alpha}\left(\int_0^{\frac{\pi}{2}}\frac{\cos^2\phi_{ring}}{\sqrt{1 - \left(1 - \frac{b^2}{a^2}\right)\sin^2\alpha\sin^2\phi_{ring}}}\dd{\phi_{ring}}\right)_{\alpha=0}\right]\\
    &= -4T \left[\frac{ab(a^2 - b^2)}{\left(a^2*1 + b^2*0\right)^{\flatfrac{3}{2}}}\right] \int_0^{\frac{\pi}{2}}\cos^2\phi_{ring}\dd{\phi_{ring}} = -4T\frac{ab(a^2-b^2)}{a^3}\frac{\pi}{4}\\
    \implies \left.\dv{\tau}{\alpha}\right|_{\alpha=0} &= -\pi T b\left(1 - \frac{b^2}{a^2}\right)
\end{align*}

Thus, by Taylor expansion, the torque to linear order in $\alpha$ is:
\begin{equation} \label{eq:torque}
    \tau \approx \left.\dv{\tau}{\alpha}\right|_{\alpha=0}\alpha =  -\pi T b\left(1 - \frac{b^2}{a^2}\right)\alpha
\end{equation}

\paragraph{Torque balance for the contractile ring}
Per our assumption of negligible inertial terms and the form of friction experienced by the ring, the torque balance of the ring can be written as:
\begin{equation*}
    \tau - \gamma\cdot\dot{\alpha} = 0
\end{equation*}
Putting in the expression for $\tau$ we just obtained, we get
\begin{equation} \label{eq:alphaDot}
    \dot{\alpha} \approx - \frac{\pi T}{\gamma} b\left[1 - \frac{b^2}{a^2}\right]\alpha
\end{equation}

\paragraph{Calculating posteriorization velocity of the male pronucleus $v$}
To get the posteriorization velocity of the male pronucleus $v$, we make an additional assumption: the angular position of the male pronucleus is the same as the angle $\alpha$ itself. Effectively, the male pronucleus acts as if it is rigidly attached to the normal vector to the contractile ring. Then, the position of the male pronucleus is given by:
\begin{align*}
    \textrm{On ellipsoid:}\quad & \frac{x_{nucl}^2}{a^2} + \frac{y_{nucl}^2}{b^2} = 1\\
    \textrm{Angle with x-axis (long axis):}\quad & x_{nucl} = r_{nucl}\cos\alpha, \quad y_{nucl} = r_{nucl}\sin\alpha\\
    \implies \frac{r_{nucl}^2\cos^2\alpha}{a^2} + \frac{r_{nucl}^2\sin^2\alpha}{b^2} &= 1\\
    \implies r_{nucl} = \frac{ab}{\sqrt{a^2\sin^2\alpha + b^2\cos^2\alpha}}
\end{align*}

The posteriorization velocity of the male pronucleus is then given by (for small angles $\alpha$, velocity is almost parallel to the cortex, hence we can take the full magnitude)
\begin{align*}
    v \approx \sqrt{\left(\dv{x_{nucl}}{t}\right)^2 + \left(\dv{y_{nucl}}{t}\right)^2} = \sqrt{\dot{r}_{nucl}^2 + r_{nucl}^2\dot{\alpha}^2} 
\end{align*}
For small angle $\alpha$, $r_{nucl} \approx a - \frac{a^3}{b^2}\alpha^2 + \mathcal{O}(\alpha^3)$ and $\dot{r}_{nucl} = -2\frac{a^3}{b^2}\alpha\dot{\alpha} + \mathcal{O}(\alpha^2)$. Thus,
\begin{equation*}
    v = \sqrt{\dot{r}_{nucl}^2 + r_{nucl}^2\dot{\alpha}^2} = \dot{\alpha}\sqrt{a^2 - 2\frac{a^4}{b^2}\alpha^2 + 4\frac{a^6}{b^4}\alpha^2 + \mathcal{O}(\alpha^3)} = \dot{\alpha}\left(a + \mathcal{O}(\alpha^2)\right)
\end{equation*}

Therefore, using \eqref{eq:alphaDot}, we get (to linear order in $\alpha$),
\begin{equation}\label{eq:velocity}
    v \approx - \left[\frac{\pi T}{\gamma}\right] \left[ab\left(1 - \frac{b^2}{a^2}\right)\right]\alpha
\end{equation}
where we select the negative sign to ensure correspondence with observed posteriorisation velocity in experiments.

\paragraph{Estimating relation between aspect ratio and posteriorization velocity}
To obtain a relation between the aspect ratio \aspectRatio and posteriorization velocity $v$ for the combined set of embryos, we use the following procedure:
\begin{enumerate}
    \item Estimate $\kappa = \left[\frac{\pi T}{\gamma}\right]$ by fitting a straight line to the posteriorization velocity versus angular position graph for the unperturbed embryos
    \item Use the average semi-minor axis length in the combined dataset to transform \eqref{eq:velocity} into a relation between \aspectRatio and $v$.
    \item Compare relation between aspect ratio and posteriorisation velocity from transformed \eqref{eq:velocity} with those from the theoretical model and from experiments.
\end{enumerate}

To estimate $\kappa = \left[\frac{\pi T}{\gamma}\right]$, we use the set comprised only of unperturbed embryos. We plot the average posteriorization velocity observed in these unperturbed embryos as a function of the angular position of the male pronucleus, and choose a set of angular positions where the posteriorization velocity looks mostly linear with respect to the angular position. We chose the angular position bins: \SIrange{3}{15}{\unitAngle}. From the slope $m$ = \SI{-3.37e-3}{\unitPostVel\per\unitAngle} of the linear fit in the selected angular position range (\SIrange{3}{15}{\unitAngle}), we can obtain $\kappa$ using \eqref{eq:velocity} as:
\begin{equation} \label{eq:slopeFit}
    m = - \left[\frac{\pi T}{\gamma}\right] \left[a_{un}b_{un}\left(1 - \frac{b_{un}^2}{a_{un}^2}\right)\right] = - \kappa \left[a_{un}b_{un}\left(1 - \frac{b_{un}^2}{a_{un}^2}\right)\right] \implies \kappa = \frac{-m}{a_{un}b_{un}\left(1 - \frac{b_{un}^2}{a_{un}^2}\right)}
\end{equation}
where the subscript $un$ refers to the average semi-major $a_{un}$ and semi-minor $b_{un}$ axes lengths for the unperturbed embryos.

From \autoref{subsec:roundEmbryosIma3}, we obtain the values of $a_{un}$ =\SI{28.7 +- 1.6}{\unitLength} and $b_{un}$ = \SI{16.2 +- 1.1}{\unitLength}. Using \eqref{eq:slopeFit}, we get the value of $\kappa$ as:
\begin{equation}\label{eq:kappa}
    \kappa = \frac{-(-3.37e-3)}{(28.7)(16.2)\left(1 - \frac{(16.2)^2}{(28.7)^2}\right)} = \SI{1.06e-5}{\per\second\per\unitAngle\per\unitLength}
\end{equation}

To then transform \eqref{eq:velocity} into a relation between \aspectRatio and $v$, we look at the distribution of semi-major \longAxisLength and semi-minor \shortAxisLength axes length in the combined set of embryos. We note that the the combined set has a wide range of \longAxisLength, but \shortAxisLength are mostly concentrated around the average value of \SI{15.6 +- 1.5}{\unitLength}. Therefore, we consider \shortAxisLength to be fairly constant throughout the combined set, allowing us to rewrite \eqref{eq:velocity} as:
\begin{equation} \label{eq:scalingRelation}
    v \approx - \left[\frac{\pi T}{\gamma}\right] \left[ab\left(1 - \frac{b^2}{a^2}\right)\right]\alpha = [\kappa b^2 \alpha] \left(\frac{a}{b}\left(1 - \frac{b^2}{a^2}\right)\right) = C \left(\frac{a}{b}\left(1 - \frac{b^2}{a^2}\right)\right)
\end{equation}
where $C = -\kappa b^2 \alpha$ is treated as a constant when considering the relation between \aspectRatio and $v$ alone. Given that the distribution of angular positions within the \SIrange{3}{15}{\unitAngle} is mostly uniform for the combined set of embryos, we set $\alpha$ = \SI{9}{\unitAngle} -- the average of the angular positions in \SIrange{3}{15}{\unitAngle}.

Therefore, $C$ can be calculated as:
\begin{equation} \label{eq:scalingConst}
    C = -\kappa b^2 \alpha = (\SI{1.06e-5}{\per\second\per\unitAngle\per\unitLength}) (\SI{15.6}{\unitLength})^2 (\SI{9}{\unitAngle}) = \SI{-2.32e-2}{\unitPostVel}
\end{equation}
Using this in \autoref{eq:scalingRelation}, we can now calculate the relation between \aspectRatio and posteriorisation velocity $v$ from the effective model. On comparison with the relation between \aspectRatio and posteriorisation velocity obtained using the effective model with that obtained from experimental observations, we find that the effective model recapitulates the experimentally observed relation to a reasonable extent. We thus conclude that our effective model -- comprising only of two features from the theoretical model: ellipsoidal geometry of the embryo and contractile ring (pseudocleavage furrow) -- does indeed capture the relation between embryo geometry and \ac{ap} axis alignment. 

%A simpler model can capture this geometry dependence. 
%Explain this effective model - contractile ring on an ellipsoid
%Calculate results for small angles
%Note that short axis is constant between unperturbed and ima-3. Use it to convert to aspect ratio relation.
%Experimental relation: Binning by aspect ratio
%Compare to experimental relation. Conclude a good fit, with some issues, that captures trend.

\section{Additional experiments}\label{sec:additionalExp}
In this section, we describe the additional experiments we conducted during this study and their results.

\subsection{Exploring relation between embryo geometry and \acs{ap} axis alignment}\label{sec:moreRounderEmbryosDpy11Ima3}

\subsubsection{Classifying \geneExp{ima-3} \ac{rnai} embryos based on aspect ratio}\label{subsubsec:splitByAspectRatios}
In \autoref{subsec:roundEmbryosIma3}, we observed that \ac{rnai} of \geneExp{ima-3} generates embryos with smaller aspect ratio (that is, rounder) compared to those observed in the unperturbed embryos. There, we inferred that embryo geometry plays a role in \ac{ap} axis alignment -- specifically, slower \ac{ap} axis alignment in rounder embryos -- by comparing the posteriorisation velocity observed in \geneExp{ima-3} \ac{rnai} embryos to those observed in unperturbed embryos, as a function of angular position. Here, we instead leverage the distribution of aspect ratios in \geneExp{ima-3} \ac{rnai} embryos -- \aspectRatio = \numrange{1.1}{1.9}. 

Specifically, we classify the \geneExp{ima-3} \ac{rnai} embryos into two categories based on the aspect ratio of each embryo: round (\aspectRatio $<$ \num{1.5}) and elliptical (\aspectRatio $>$ \num{1.5}). Measuring the semi-major and semi-minor axes length in each category, we find the average axes lengths for the round \geneExp{ima-3} \ac{rnai} embryos as \longAxisLength = \SI{20.9 +- 2.2}{\unitLength}, \shortAxisLength = \SI{15.35 +- 2.2}{\unitLength}; and for the elliptical 
\geneExp{ima-3} \ac{rnai} embryos as \longAxisLength = \SI{20.9 +- 2.2}{\unitLength}, \shortAxisLength = \SI{15.35 +- 2.2}{\unitLength}. Interestingly, we find that the average volume of the embryos are not so different between the two categories: \SI{88.88 +- 8.8}{\unitLength\cubed} in round embryos compared to \SI{88.88 +- 8.8}{\unitLength\cubed} in elliptical embryos. The volume for each embryo is calculated using the length of long ($2a_{emb}$) and short ($2b_{emb}$) axes measured for that embryo -- its volume is calculated as the volume of an ellipsoid with a long axis of the same length as the long axis of the embryo and two equal short axes of the same length as the short axis of the embryo (volume = $\flatfrac{4\pi}{3}$ $a_{emb}b_{emb}^2$). This is in contrast with what was observed for the unperturbed and \geneExp{ima-3} \ac{rnai} embryos in \autoref{subsec:roundEmbryosIma3} -- there, we observed that unperturbed embryos have a larger volume compared to \geneExp{ima-3} \ac{rnai} embryos. We also compared cortical flows between the two categories, finding that cortical flows are comparable between the two categories of \geneExp{ima-3} \ac{rnai} embryos. Specifically, we observe average cortical speed of \SI{4.18 +- 0.2}{\unitCrtxVel} for round embryos compared to \SI{3.88 +- 0.2}{\unitCrtxVel}. We next compare the average posteriorisation velocity as a function of angular position observed in the round embryos to those observed in the elliptical embryos. We find that in general round embryos exhibit slower average posteriorisation velocity compared to those observed in elliptical embryos, with larger differences at higher angular positions. Specifically, we observe that round and elliptical embryos exhibit similar average posteriorisation velocity for angular positions upto \SI{9}{\unitAngle}; for higher angular positions, round embryos have slower average posteriorisation velocity compared to elliptical ones. However, note that the difference between the average posteriorisation velocity observed in round and elliptical embryos is not as large as that observed between the \geneExp{ima-3} \ac{rnai} embryos (as a whole) and unperturbed embryos.

Altogether, these observations lead us to reconfirm the result from \autoref{subsec:roundEmbryosIma3} -- rounder embryos indeed show diminished \ac{ap} axis alignment. Unlike the case considered in \autoref{subsec:roundEmbryosIma3} however, here the two sets of embryos we consider have similar volumes but different aspect ratios.

\subsubsection{\acs{ap} axis alignment in round embryos generated in \geneExp{dpy-11} mutant}\label{subsubsec:dpy11Mutant}
In \autoref{subsec:roundEmbryosIma3}, we generated round embryos by performing a \ac{rnai} of \geneExp{ima-3}. Such round embryos can also be generated by short worms mutant in \geneExp{dpy-11} \citep{ko2002novel,yamamoto2017asymmetric}. DPY-11 is involved in the development of the cuticle of worms during post-embryonic morphogenesis, and plays an important role in the molting cycle of the worms \citep{ko2002novel}. \geneExp{dpy-11} mutant worms are shorter than unperturbed worms. In this section, we use the SWG297 strain -- a strain of worms mutant in \geneExp{dpy-11} tagged with \flurophoreLabel{\ac{nmy2}}{\ac{gfp}} and \flurophoreLabel{PHDomain}{mCherry} (see \autoref{sec:wormHandling} for details).

Here, we describe the observed \ac{ap} axis alignment in three different sets of embryos:
\begin{itemize}
    \item Embryos generated from the SWG297 strain. We call these \geneExp{dpy-11} mutant embryos
    \item Embryos generated by performed a \ac{rnai} of \geneExp{ima-3} on worms of SWG297 strain, for a feeding time of \SI{24}{\unitRNAiTime}. We call these \geneExp{ima-3} \ac{rnai}, \geneExp{dpy-11} mutant embryos.
    \item Embryos generated by performed a double \ac{rnai} of \geneExp{ima-3} and \geneExp{mel-11} on worms of SWG297 strain, for a feeding time of \SI{24}{\unitRNAiTime}. We call these \geneExp{ima-3; mel-11} \ac{rnai}, \geneExp{dpy-11} mutant embryos.
\end{itemize}
We first compare cortical flows measured in the three sets. We find that average cortical flow speeds are not significantly different between the three sets: \SI{4.12 +- 0.59}{\unitCrtxVel} for \geneExp{dpy-11} mutant embryos; \SI{4.12 +- 0.59}{\unitCrtxVel} for \geneExp{ima-3} \ac{rnai}, \geneExp{dpy-11} mutant embryos; \SI{4.12 +- 0.59}{\unitCrtxVel} for \geneExp{ima-3; mel-11} \ac{rnai}, \geneExp{dpy-11} mutant embryos. Given that MEL-11 reduces the amount of myosin on the cortex (as discussed in \autoref{subsec:Nop1AndNop1Mel11}), this observation is surprising in that cortical flows are not faster in \geneExp{ima-3; mel-11} \ac{rnai}, \geneExp{dpy-11} mutant embryos. However, this could also be because of low number of embryos sampled for this set. In any case, we observe that cortical flows are similar between the three sets of embryos considered here. Note that since the cortical flows observed here are slower than those observed in unperturbed embryos or \geneExp{ima-3} \ac{rnai} embryos, a direct comparison with those sets of embryos is not feasible.

We next compare the axes lengths and aspect ratios observed in the three sets. We observe that \geneExp{ima-3; mel-11} \ac{rnai}, \geneExp{dpy-11} mutant embryos are the most round, with average aspect ratio \aspectRatio = \num{1.2 +- 0.2} and average axes lengths \longAxisLength = \SI{10.00 +- 0.00}{\unitLength}, \shortAxisLength = \SI{10.00 +- 0.00}{\unitLength}; \geneExp{dpy-11} mutant embryos are most elliptical, with average aspect ratio \aspectRatio = \num{1.2 +- 0.2} and average axes lengths \longAxisLength = \SI{10.00 +- 0.00}{\unitLength}, \shortAxisLength = \SI{10.00 +- 0.00}{\unitLength}; and \geneExp{ima-3} \ac{rnai}, \geneExp{dpy-11} mutant embryos are in the middle, with average aspect ratio \aspectRatio = \num{1.2 +- 0.2} and average axes lengths \longAxisLength = \SI{10.00 +- 0.00}{\unitLength}, \shortAxisLength = \SI{10.00 +- 0.00}{\unitLength}. Thus, we have a gradation of aspect ratios between the three sets of embryos.

Finally, we compare posteriorisation velocity as a function of angular positions between the three set of embryos. We find that for all angular positions, \geneExp{ima-3; mel-11} \ac{rnai}, \geneExp{dpy-11} mutant embryos exhibit the slowest posteriorisation velocity. Additionally, posteriorisation velocity is comparable between \geneExp{ima-3} \ac{rnai}, \geneExp{dpy-11}  mutant embryos and \geneExp{dpy-11}  mutant embryos, with slightly faster posteriorisation velocity observed in the latter --  upto angular position \SI{15}{\unitAngle}. Betweeen angular positions \SI{15}{\unitAngle} and \SI{21}{\unitAngle}, \geneExp{dpy-11}  mutant embryos seem to show slower posteriorisation velocity comparable to those observed in \geneExp{ima-3; mel-11} \ac{rnai}, \geneExp{dpy-11} mutant embryos. We believe this is because of the low number of embryos present in the \geneExp{dpy-11} mutant embryos set -- specifically, there is only one embryo, with aspect ratio \aspectRatio = \num{1.2}, contributing to angular position bins between \SI{15}{\unitAngle} and \SI{21}{\unitAngle} (comparable to aspect ratios observed in the \geneExp{ima-3; mel-11} \ac{rnai}, \geneExp{dpy-11} mutant embryos set).

The observations here generally match the qualitative behaviour expected from the relation between aspect ratio and posteriorisation velocity discussed in \autoref{sec:GeometryRole}. Given the low number of embryos in each set discussed here, such a match is only indicative of the relation between embryo geometry and \ac{ap} axis alignment.

\subsection{Are pseudocleavage furrow-dependent and cytoplasmic flow-dependent mechanisms sufficient to explain \acs{ap} axis alignment?}\label{subsec:sufficiencyTestAir1}

\subsubsection{Decoupling positions of male pronucleus and polarisation trigger via \geneExp{air-1} \ac{rnai}}\label{subsubsec:air1rnaiDecoupleMalePronucleusPolarisation}
In \autoref{sec:ApAxisEstablishment}, we discussed the mechanism of \ac{ap} axis establishment, and the role of polarisation trigger provided by the centrosomes anchored to the male pronucleus in this process. Given this anchoring \citep{de2016dynein}, we have until now used the position of the male pronucleus as a proxy to the position of the polarisation trigger, and thus the instantaneous orientation of the \ac{ap} axis. We here investigate if this link between the position of the male pronucleus and polarisation trigger could be broken, and the effect on \ac{ap} axis alignment.

To do so, we perform an \ac{rnai} of \geneExp{air-1} on worms of SWG070 strain, for a feeding time of \SI{24}{\unitRNAiTime} (see \autoref{sec:rnaiMethods} for details on \ac{rnai}). AIR-1 (or Aurora Kinase A) is a serine/theronine kinase that is essential for centrosome maturation \citep{hannak2001aurora}, among other functions related to mitotis and cytokinesis \citep{kapoor2019centrosome}. Previous studies have observed that \ac{rnai}-mediated depletion of \geneExp{air-1} leads to embryos with mutliple \ac{ppar} domains instead of a single one observed in unperturbed embryos, along with two pseudocleavage furrows \citep{kapoor2019centrosome,klinkert2019aurora}. Additionally, in these \ac{rnai} embryos the centrosome is dispensible for positioning of the \ac{ap} axis \citep{kapoor2019centrosome,klinkert2019aurora}. 

We find that our observations are consistent with those made in \cite{klinkert2019aurora} and \cite{kapoor2019centrosome}. Specifically, we find that \geneExp{air-1} \ac{rnai} embryos typically exhibit two pseudocleavage furrows, alongwith two domains where myosin is depleted -- representing the two \geneExp{ppar} domains. Interestingly, we find that the two myosin depletion domains form on the opposite ends of the embryo -- that is, always aligned along the long axis of the embryo. We also observe slower cortical flows in the \ac{rnai} embryos compared to those observed in unperturbed embryos: average cortical flow speed in \ac{rnai} embryos is \SI{4.12 +- 0.59}{\unitCrtxVel} compared to \SI{4.12 +- 0.59}{\unitCrtxVel} in unperturbed embryos. However, we observe no decay towards \SI{0}{\unitAngle} in the angular position of the male pronucleus with time, and very slow posteriorisation velocities for all angular positions. Additionally, we observe that the male pronucleus is typically further away from the cortex than in unperturbed embryos: \SI{10.00 +- 0.00}{\unitLength} in \ac{rnai} embryos compared to \SI{10.00 +- 0.00}{\unitLength} in unperturbed embryos. We thus conclude that the male pronucleus does not posteriorise in the \geneExp{air-1} \ac{rnai} embryos, indicating that the \ac{rnai} embryos do not need the male pronucleus to \enquote{guide} the positioning of \ac{ppar} domain(s) (guidance here refers to centrosomes anchored to the male pronucleus -- see \cite{gross2019guiding}). This is consistent with the results of \cite{klinkert2019aurora} and \cite{kapoor2019centrosome}, which show that the centrosome is dispensible for positioning of the \ac{ap} axis in \geneExp{air-1} \ac{rnai} embryos. However, even with no guidance from the centrosomes, we observe that the polarisation in \geneExp{air-1} \ac{rnai} embryos aligns with the long axis of the embryo.

We note that in the theoretical model, the male pronucleus is advected by flows in the cytoplasm -- see \autoref{ch:ActiveMatter}. If however the male pronucleus is no longer coupled to the polarisation trigger, as seems to be case in \geneExp{air-1} \ac{rnai} embryos, then we expect that the advection of the male pronucleus by cytoplasmic flows should have no effect on embryo polarisation. Thus, in \geneExp{air-1} \ac{rnai} embryos, cytoplasmic flow-dependent mechanism is unlikely to play a role in the alignment of the \ac{ap} axis observed in \geneExp{air-1} \ac{rnai} embryos. The pseudocleavage furrow-dependent mechanism enforces a rotation in the cortex itself, and thus should be less impacted by the broken coupling between the male pronucleus and polarisation trigger in \geneExp{air-1}
\ac{rnai} embryos. Thus, \geneExp{air-1} \ac{rnai} may be a possible genetic perturbation to disable the cytoplasmic flow-dependent mechanism for \ac{ap} axis alignment.

%Remind reader of the link between male pronucleus and polarisation trigger. Describe how air-1 breaks this by making par system critical. Describe polarisation in air-1 embryos - double \ac{ppar} domain, but still aligned along the long axis (needs statistics); double pseudocleavage furrow. Show that posteriorisation of the male pronucleus does not occur, show that cortical flows still exist, show that male pronucleus randomly moves around in the cytoplasm. Conclude that this indicates the cortex is capable of aligning by itself once trigger is provided -- without guidance from male pronucleus. This is expected from the ring-like model of the pseudocleavage furrow-dependent mechanism, and not expected from the cytoplasmic flow-dependent mechanism.

%Remind reader that the model requires the male pronucleus to transmit the flows in the cytoplasm to the cortex. Note that here we have broken that coupling, still alignment is fine. Conclude that this implies the cytoplasmic flow-dependent mechanism does not play a role in air-1 embryos. Show example of movie in air-1 where the cytoplasmic flow do not follow nucleus at all, instead following the moving depletion. 

\subsubsection{Suppressing both mechanisms via \geneExp{air-1; mel-11} \ac{rnai} in \geneExp{nop-1} mutant}
To test if the \ac{ap} axis still aligns with the long axis after both the cytoplasmic-flow dependent and pseudocleavage furrow-dependent mechanism are disabled, we perform a double \ac{rnai} of \geneExp{air-1; mel-11} on \geneExp{nop-1} mutant worms from the SWG228 strain, for a feeding time of \SI{24}{\unitRNAiTime} (see \autoref{sec:rnaiMethods} for details on \ac{rnai}). This combines the two conditions we considered before: \ac{rnai} of \geneExp{air-1} to disable cytoplasmic flow-dependent mechanism, and \ac{rnai} of \geneExp{mel-11} in \geneExp{nop-1} mutant background to disable the pseudocleavage furrow-dependent mechanism. \geneExp{nop-1} mutant worms were chosen due to the difficulty in performing a triple RNAi condition and since \geneExp{nop-1} mutation is not lethal.

As expected from observations in \geneExp{air-1} \ac{rnai} embryos, the male pronucleus in the \geneExp{air-1; mel-11} \ac{rnai} in \geneExp{nop-1} mutant embryos is observed to be further away from the cortex than in unperturbed embryos: \SI{10.00 +- 0.00}{\unitLength} in \ac{rnai} embryos compared to \SI{10.00 +- 0.00}{\unitLength} in unperturbed embryos. As expected from observations in \geneExp{nop-1; mel-11} \ac{rnai} embryos, no pseudocleavage furrows are observed in \geneExp{air-1; mel-11} \ac{rnai} in \geneExp{nop-1} mutant embryos. However, we observe that the domain where myosin is depleted (indicating the \ac{ppar} domain) does not re-align back towards the tip of the embryo, in the few embryos that start with a myosin depletion domain away from the poles of the embryos. These preliminary experiments seem to suggest that \ac{ap} axis alignment is heavily suppressed in \geneExp{air-1; mel-11} \ac{rnai} in \geneExp{nop-1} mutant embryos -- however, additional experimental data is needed for a conclusive result.

%Remind reader that nop-1/mel-11 rnai was used to remove the pseudocleavage furrow-dependent mechanism. Use the same idea here -  describe the difference: triple rnai is difficult to do, so instead we use double rnai of air1/mel11 in nop1 mutant background. Describe the condition - male pronucleus not near cortex, no pseudocleavage furrow. Not too many embryos with high angle obtained. Show movie with high angle as it attempts to align, but fails. Indicate if this observation was seen all time (numbers). Preliminary conclusion: suppressing both mechanisms seems to lead to almost no alignment, indicating both mechanisms capture behaviour. Speculate if observed movement of depletion domain is due to mechanism described in erwin frey paper.

\subsection{Role of microtubules in \acs{ap} axis alignment}\label{subsec:MicrotubuleRoleGoa1Gpa16}
In \autoref{sec:ApAxisEstablishment}, we discussed the mechanism of \ac{ap} axis establishment. Specifically, we described the important role of the centrosomes attached to the male pronucleus in providing the polarity triggers required for initiating \ac{ap} axis establishment. In \autoref{subsec:ComponentsCytoskeleton}, we also described the role of centrosomes as nucleators for microtubules, which emanate as an aster from the centrosomes. Additionally, these microtubules play an important role in stabilizing the nascent \ac{ppar} domain that forms during \ac{ap} axis establishment (see \autoref{sec:ApAxisEstablishment} and \cite{wallenfang2000polarization}). 

In our investigation of the mechanism(s) driving \ac{ap} axis alignment so far, we have neglected any role of the astral microtubules emanating from the centrosomes. Specifically, we ask if dynein motors anchored at the cortex could play a role in \ac{ap} axis alignment, by pulling onto the astral microtubules abutting the cortex. As discussed in \autoref{subsec:ComponentsCytoskeleton}, dynein motors are one of two families of motor proteins that bind to microtubules. They are (-)-end directed motor proteins \citep{de2016dynein}. Previous studies have explored the role of dynein motors in the positioning of centrosomes. Especially, dynein motors anchored to the cortex have been found to be influential in positioning the mitotic spindle, by pulling onto astral microtubules \citep{de2016dynein,gotta2001distinct,nguyen2007coupling,colombo2003translation}. We here investigate if such pulling forces generated at the cortex, by said dynein motors, could influence the posteriorisation of the male pronucleus observed during \ac{ap} axis alignment.

In this subsection, we use embryos from the SWG057 strain instead of the SWG070 strain we have been using so far. SWG057 strain is labelled with \flurophoreLabel{TUB}{\ac{gfp}} -- which labels tubulin monomers that form the microtubules -- and \flurophoreLabel{\ac{nmy2}}{mKate}. Thus this strain allows us to visualise both the microtubules and the centrosomes, and cortical myosin -- while still allowing us to track the position of the male pronucleus as a dark circle in the myosin channel -- see \autoref{sec:imageAnalysis} for details on movies generated with embryos from SWG057 were analysed. We take time-lapse microscopy of one-cell stage embryos generated from SWG057 strain -- see \autoref{sec:microscope} for details on microscopy. 

We characterise the rate of \ac{ap} axis alignment in unperturbed embryos of the SWG057 strain, by measuring the posteriorisation velocity of the male pronucleus as a function of angular position of the male pronucleus in these embryos. We observe that in these unperturbed embryos the male pronucleus is, on average, moving towards the posterior end -- with higher magnitude of posteriorisation velocity at higher angular positions. We also measure the cortical flows in unperturbed embryos from SWG057 strain -- see \autoref{sec:imageAnalysis} for methods. We observe an average cortical speed of \SI{4.12 +- 0.59}{\unitCrtxVel}. We also bin the observed cortical flows using angular positions of the male pronucleus -- see \autoref{sec:statAnalysis}. We find that the point where the cortical flows change sign correlates with the angular position. Given that these observations are similar to those made for the unperturbed embryos from the SWG070 strain (see \autoref{sec:apAxisAlignCharacteriseWT}), we conclude that the unperturbed embryos from SWG057 exhibit similar \ac{ap} axis alignment characteristics as compared to unperturbed embryos from SWG070.

To test if cortically anchored dynein motors pulling on astral microtubules influence \ac{ap} axis alignment, we suppress the proteins that anchor the dynein motors to the cortex. Specifically, we perform a double \ac{rnai} of \geneExp{goa-1} and \geneExp{gpa-16} on worms of SWG057 strain, for a feeding time of \SI{24}{\unitRNAiTime} (see \autoref{sec:rnaiMethods} for details on double \ac{rnai}). GOA-1 and GPA-16 are heterotrimeric G$\alpha$ proteins that, along with GPR-1/2 and LIN-5, help anchor dynein at the cortex \citep{}. Thus, their reduction via the double \ac{rnai} leads to depletion of cortically anchored dynein motors. We measure the distance between the centrosomes -- identifiable as bright white spots in the tubulin channel. We observe that the distance between centrosomes, plotted as a function of time, grows slower in the \geneExp{goa-1; gpa-16} \ac{rnai} embryos compared to that in unperturbed embryos. This is in accordance with observations made in \cite{}, thus confirming that the double \ac{rnai} does reduce cortically anchored dynein. Next, we compare the cortical flows measured in the double \ac{rnai} embryos with those observed in unperturbed embryos. We find that the average cortical speeds are not significantly different between the double \ac{rnai} embryos and unperturbed embryos: \SI{4.12 +- 0.59}{\unitCrtxVel} in double \ac{rnai} embryos compared to \SI{4.12 +- 0.59}{\unitCrtxVel} in unperturbed embryos. Finally, we compare the posteriorisation velocity observed in double \ac{rnai} embryos to those observed in unperturbed embryos, for different angular positions. We again find no significant difference between the two, for all angular positions considered (\SIrange{0}{24}{\unitAngle}) -- indicating that the posteriorisation of the male pronucleus does not differ significantly between the double \ac{rnai} embryos and unperturbed embryos. These observations indicate that the double \ac{rnai} \geneExp{goa-1; gpa-16} does not significantly affect either the cortical flow or the posteriorisation of the male pronucleus. We thus conclude that forces generated by cortical dynein as they pull on astral microtubule do not play an significant role in \ac{ap} axis alignment.